\documentclass[10pt]{article}

% Paketi koji će se koristiti
\usepackage[a4paper, landscape, margin=2.00cm]{geometry}
\usepackage{amsfonts, amssymb, amsmath}
\usepackage{graphicx}
\usepackage{float}
\usepackage{multirow}
\usepackage{array}
\usepackage[utf8]{inputenc}
\usepackage[T1]{fontenc}
\usepackage{fancyhdr}
\usepackage{lastpage}
\usepackage[dvipsnames]{xcolor}
\usepackage{pifont}
\usepackage{hyperref}
\hypersetup{
    colorlinks=true,
    linkcolor=blue,
    filecolor=magenta,      
    urlcolor=cyan,
    pdftitle={python},
    pdfpagemode=FullScreen,
}

% Dodaj header i footer
\pagestyle{fancy}
\renewcommand{\headrulewidth}{0pt}
\fancyhf{}
\setlength{\headheight}{32.25pt}
\lhead{}
\rhead{\includegraphics[height=1.00cm]{./python.png}}
\setlength{\footskip}{12.00pt}
\lfoot{}
\rfoot{str. {\thepage}/{\pageref{LastPage}}}

\begin{document}

\renewcommand{\arraystretch}{1.50}

    \section*{\color{NavyBlue} Komentiranje}
    \begin{tabular}{|>{\tt}p{9.00cm}|>{}p{15.50cm}|}
        \hline
        \# & početak linijskog komentara \\ \hline
    \end{tabular}

    \section*{\color{NavyBlue} Osnovni tipovi varijabli / objekata}
    \begin{tabular}{|>{\tt}p{9.00cm}|>{\tt}p{0.25cm}|>{}p{14.75cm}|}
        \hline
        str & \ding{55} & (string), (uređeni) niz znakova
        \\ \hline
        int & \ding{55} & (integer), cijeli broj
        \\ \hline
        float & \ding{55} & broj s pomičnim zarezom
        \\ \hline
        bool & \ding{55} & (boolean), logička varijabla \textbf{True} ili \textbf{False}
        \\ \hline
    \end{tabular}
    \begin{center}
        \ding{55} - objekt \textbf{nije identiteno-promjenjiv} tj. \textit{immutable} je
        \\
        \ding{51} - objekt \textbf{je identiteno-promjenjiv} tj. \textit{mutable} je
        \\
    \end{center}

    \section*{\color{NavyBlue} Složeni tipovi varijabli / objekata}
    \begin{tabular}{|>{\tt}p{9.00cm}|>{\tt}p{0.25cm}|>{}p{14.75cm}|}
        \hline
        range & \ding{55} & raspon - uređen (indeksiran) niz integera
        \\ \hline
        list & \ding{51} & lista - uređena (indeksirana) podatkovna kolekcija (niz objekata)
        \\ \hline
        dict & \ding{51} & rječnik - neuređena (neindeksirana) podatkovna kolekcija parova ključ-vrijednost
        \\ \hline
        tuple & \ding{55} & n-terac - uređena (indeksirana) \textbf{nepromjenljiva} podatkovna kolekcija
        \\ \hline
        set & \ding{55} & skup - neuređena (neindeksirana) podatkovna kolekcija \textbf{jedinstvenih} vrijednosti
        \\ \hline
    \end{tabular}
    \begin{center} 
        \ding{55} - objekt \textbf{nije promjenjiv} tj. \textit{immutable} je
        \\
        \ding{51} - objekt \textbf{je promjenjiv} tj. \textit{mutable} je
        \\
    \end{center}

    \section*{\color{NavyBlue} Konverzije varijabli}
    \begin{tabular}{|>{\tt}p{9.00cm}|>{}p{15.50cm}|}
        \hline
        str(object)     & konverzija u string  \\ \hline
        int(object)     & konverzija u integer \\ \hline
        float(object)   & konverzija u float   \\ \hline
        bool(object)    & konverzija u boolean \\ \hline
        list(iterable)  & konverzija (podatke kolekcije) u listu \\ \hline
    \end{tabular}

    \section*{\color{NavyBlue} Funkcije s objektima}
    \begin{tabular}{|>{\tt}p{9.00cm}|>{}p{15.50cm}|}
        \hline
        len(\textit{object}) & vraća broj znakova u objektu
        \\ \hline
        print(\textit{object}) & ispisuje objekt (ne vraća ništa)
        \\ \hline
        type(\textit{object}) & ispisuje tip objekta
        \\ \hline
        id(\textit{object}) & ispisuje identitet objekta (adresu u memoriji) *
        \\ \hline
    \end{tabular}
    \begin{center}
        \footnotesize
        * Svi objekti u Pythonu imaju svoj jedinstveni ID, koji se dodjeljuje objektu kada se objekt \textbf{kreira}. \\
        ID objekta je adresa u memoriji i bit će drugačija svaki put kad se pokrene program. \\
        Objekti kojima se vrijednosti mogu mijenjati bez promjene identiteta zovu se promjenljivi (engl. mutable) objekti, a oni kojima se vrijednost ne može mijenjati bez stvaranja novog objekta istog tipa zovu se nepromjenljivi (engl. immutable) objekti. \\
        Promjena vrijednosti objekta obično se događa operatorom pridružbe ili djelovanjem metode na vrijednost objekta. \\
    \end{center}

    \section*{\color{NavyBlue} Ostale funkcije}
    \begin{tabular}{|>{\tt}p{9.00cm}|>{}p{15.50cm}|}
        \hline
        range([start], end, [step]) & vraća sekvencu (raspon) brojeva od "start" do "end", s razlikom od "step"
        \\ \hline
    \end{tabular}

    \section*{\color{NavyBlue} Deklaracija stringova}
    \begin{tabular}{|>{\tt}p{9.00cm}|>{}p{15.50cm}|}
        \hline
        "text"                                  & klasičan način zapisa stringa                                            \\ \hline
        """text"""                              & zapis stringa koji omogućuje prijelaz u novi red                         \\ \hline
        f"text \{\textit{value}\} text"         & formatirani string koji evaluira vrijednost u vitičastim zagradama       \\ \hline
    \end{tabular}
    \begin{center}
        \footnotesize
        \textbf{Prilikom deklaracije, mogu se koristiti i jednostruki navodnici!}
        Stringovi nisu identiteno-promijenjivi... \\
        \texttt{>> string1 = "Ivan"} \\
        \texttt{>> string2 = string1} \\
        \texttt{>> id(string1) == id(string2)} \\
        \texttt{"True"} \\
        \texttt{>> string1 = "Marko"} \\
        \texttt{>> string2} \\
        \texttt{"Ivan"} \\
        \texttt{>> id(string1) == id(string2)} \\
        \texttt{"False"} \\
        ... što znači da promjena jedne ne veže promjenu druge.
    \end{center}
    
    \section*{\color{NavyBlue} Osnovne operacije sa stringovima}
    \begin{tabular}{|>{\tt}p{9.00cm}|>{}p{15.50cm}|}
        \hline
        "text"[0] & indeksiranje stringova (\texttt{"t"})
        \\ \hline
        "text"[0:2:1] & komad stringa (počevši od 0 do 2 (ne uključujući 2), s korakom 1) (\texttt{"te"})
        \\ \hline
        "abc" + "def" & povezivanje stringova (\texttt{"abcdef"})
        \\ \hline
        "abc" * 2 & umnožavanje stringova (\texttt{"abcabc"})
        \\ \hline
    \end{tabular}
    \begin{center}
        \footnotesize
        Posljednji član stringa ima indeks -1, predzadnji -2, itd.
    \end{center}

    \section*{\color{NavyBlue} Iteriranje kroz stringove}
    \begin{tabular}{|>{\tt}p{9.00cm}|>{}p{15.50cm}|}
        \hline
        for \textit{char} in \textit{string}: & iteriranje kroz string
        \\ \hline
    \end{tabular}
    
    \section*{\color{NavyBlue} Funkcije sa stringovima}
    \begin{tabular}{|>{\tt}p{9.00cm}|>{}p{15.50cm}|}
        \hline
        input(\textit{prompt}) & traži vrijednost inputa u prompt (tipa string)
        \\ \hline
    \end{tabular}

    \section*{\color{NavyBlue} Metode nad stringovima}
    \begin{tabular}{|>{\tt}p{9.00cm}|>{\tt}p{0.25cm}|>{}p{14.50cm}|}
        \hline
        \textit{string}.isnumeric() & \ding{51} & ispisuje \textbf{True} ako je string broj, a \textbf{False} ako nije
        \\ \hline
        \textit{string}.isalpha() & \ding{51} & ispisuje \textbf{True} ako je string sastavljen isključivo od slova, a \textbf{False} ako ne
        \\ \hline
        \textit{string}.find(\textit{value}, \textit{[start]}, \textit{[end]}) & \ding{51} & vraća indeks prvog pojavljivanja "value"
        \\ \hline
        \textit{string}.index(\textit{value}, \textit{[start]}, \textit{[end]}) & \ding{51} & slično kao find, ali u slučaju nepronalaska izbacuje grešku
        \\ \hline
        \textit{string}.rfind(\textit{value}, \textit{[start]}, \textit{[end]}) & \ding{51} & vraća indeks zadnjeg pojavljivanja "value"
        \\ \hline
        \textit{string}.rindex(\textit{value}, \textit{[start]}, \textit{[end]}) & \ding{51} & slično kao rfind, ali u slučaju nepronalaska izbacuje grešku
        \\ \hline
        \textit{string}.lstrip(\textit{[character]}) & \ding{51} & uklanja niz "character" (ili razmak) s lijeve strane \textbf{kopije} stringa
        \\ \hline
        \textit{string}.rstrip(\textit{[character]}) & \ding{51} & uklanja niz "character" (ili razmak) s desne strane \textbf{kopije} stringa
        \\ \hline
        \textit{string}.strip(\textit{[character]}) & \ding{51} & uklanja niz "character" (ili razmak) s lijeve i desne strane \textbf{kopije} stringa
        \\ \hline
        \textit{string}.replace(\textit{oldvalue}, \textit{newvalue}, \textit{[count]}) & \ding{51} & mijenja "oldvalue" za "newvalue" (u prvih "count" pojavljivanja) \textbf{kopije} stringa
        \\ \hline
        \textit{string}.count(\textit{value}, \textit{[start]}, \textit{[end]}) & \ding{51} & ispisuje broj ponavljanja "value"
        \\ \hline
        \textit{string}.lower() & \ding{51} & ispisuje \textbf{kopiju} stringa kojemu su sva slova mala
        \\ \hline
        \textit{string}.upper() & \ding{51} & ispisuje \textbf{kopiju} stringa kojemu su sva slova velika
        \\ \hline
        \textit{string}.capitalize() & \ding{51} & ispisuje \textbf{kopiju} stringa kojemu je prvo slovo veliko
        \\ \hline
        \textit{string}.split(\textit{delimiter}) & \ding{51} & ispisuje \textbf{novu listu} nastalu razdvajanjem \textbf{kopije stringa} po graničniku
        \\ \hline
        \textit{delimiter}.join(\textit{iterable}) & \ding{51} & ispisuje \textbf{novi string} nastao spajanjem elemenata \textbf{kopije iterabla} po graničniku
        \\ \hline
    \end{tabular}
    \begin{center}
        \ding{55} - metoda \textbf{nema} povrat tj. \texttt{return}
        \\
        \ding{51} - metoda \textbf{ima} povrat tj. \texttt{return}
        \\
    \end{center}

    \section*{\color{NavyBlue} Aritmetički operatori}
    \begin{tabular}{|>{\tt}p{9.00cm}|>{}p{15.50cm}|}
        \hline
        +  & zbrajanje                      \\ \hline
        -  & oduzimanje                     \\ \hline
        *  & množenje                       \\ \hline
        ** & potenciranje                   \\ \hline
        /  & dijeljenje                     \\ \hline
        // & cjelobrojno dijeljenje         \\ \hline
        \% & ostatak cjelobrojnog djeljenja \\ \hline
    \end{tabular}

    \section*{\color{NavyBlue} Operatori dodjele}
    \begin{tabular}{|>{\tt}p{9.00cm}|>{}p{15.50cm}|}
        \hline
        =   & a = 5             \\ \hline
        +=  & a = a + 5         \\ \hline
        -=  & a = a - 5         \\ \hline
        *=  & a = a * 5         \\ \hline
        **= & a = a ** 5        \\ \hline
        /=  & a = a / 5         \\ \hline
        //= & a = a // 5        \\ \hline
        \%= & a = a \% 5        \\ \hline
    \end{tabular}

    \section*{\color{NavyBlue} Operatori usporedbe vrijednosti}
    \begin{tabular}{|>{\tt}p{9.00cm}|>{}p{15.50cm}|}
        \hline
        ==      & jednako           \\ \hline
        !=      & nije jednako      \\ \hline
        >       & veće              \\ \hline
        >=      & veće ili jednako  \\ \hline
        <       & manje             \\ \hline
        <=      & manje ili jednako \\ \hline
    \end{tabular}
    \begin{center}
        \footnotesize
        Izlaz može biti \texttt{True} ili \texttt{False}
    \end{center}

    \section*{\color{NavyBlue} Operatori usporedbe adresa u memoriji}
    \begin{tabular}{|>{\tt}p{9.00cm}|>{}p{15.50cm}|}
        \hline
        is      & jednakost adresa u memoriji \\ \hline
        is not  & nejednakost adresa u memoriji \\ \hline
    \end{tabular}
    \begin{center}
        \footnotesize
        Operator provjerava \texttt{id(object1) == id(object2)} \\
        Izlaz može biti \texttt{True} ili \texttt{False}
    \end{center}

    \section*{\color{NavyBlue} Logički operatori}
    \begin{tabular}{|>{\tt}p{9.00cm}|>{}p{15.50cm}|}
        \hline
        and   & istinito ako su sve tvrdnje točne           \\ \hline
        or    & istinito ako je barem jedna tvrdnja točna   \\ \hline
        not   & inverzija (negacija) istinitosti tvrdnje    \\ \hline
    \end{tabular}
    \begin{center}
        \footnotesize
        Izlaz može biti \texttt{True} ili \texttt{False}
    \end{center}

    \section*{\color{NavyBlue} Operatori članstva}
    \begin{tabular}{|>{\tt}p{9.00cm}|>{}p{15.50cm}|}
        \hline
        in        & točnost postojanja člana u sekvenci         \\ \hline
        not in    & netočnost postojanja člana u sekvenci       \\ \hline
    \end{tabular}
    \begin{center}
        \footnotesize
        Izlaz može biti \texttt{True} ili \texttt{False}
    \end{center}

    \section*{\color{NavyBlue} Neistinite (lažne) vrijednosti}
    \begin{tabular}{|>{\tt}p{9.00cm}|>{}p{15.50cm}|}
        \hline
        False & definicijska "laž"                                      \\ \hline
        None  & "vrijednost" varijable bez vrijednosti (a = None)       \\ \hline                            
        0     & 0 tipa cijelog broja                                    \\ \hline
        0.0   & 0 tipa broja s pomičnim zarezom                         \\ \hline
        ""    & prazni string (u bilo kojem formatu)                    \\ \hline
        []    & prazan niz                                              \\ \hline
        ()    & prazna n-torka                                          \\ \hline
        \{\}  & prazan rječnik                                          \\ \hline
        set() & prazan skup                                             \\ \hline
        range(0) & prazan raspon                                        \\ \hline
    \end{tabular}

    \section*{\color{NavyBlue} Grananje}
    \begin{tabular}{|>{\tt}p{9.00cm}|>{}p{15.50cm}|}
        \hline
        if \textit{<condition 1>}:                   & postavljanje prvog uvjeta                                         \\ 
        \hspace{5mm}<code block 1>                   & kod koji se izvršava ako je prvi uvjet zadovoljen                 \\                             
        elif \textit{<condition 2>}:                 & postavljanje drugog uvjeta                                        \\ 
        \hspace{5mm}\textit{<code block 2}>          & kod koji se izvršava ako je drugi uvjet zadovoljen                \\ 
        else:                                        & pokrivanje svih ostalih uvjeta                                    \\ 
        \hspace{5mm}\textit{<code block 3}>          & kod koji se izvršava ako prvi i drugi uvjet nisu zadovoljeni      \\ \hline
    \end{tabular}
    \begin{center}
        \footnotesize
        \texttt{if}, \texttt{elif} i \texttt{else} moraju koristiti iste indentacije! \\
        \texttt{<code block 1>}, \texttt{<code block 2>} i \texttt{<code block 3>} moraju koristiti iste indentacije!
    \end{center}
    \begin{center}
        \footnotesize
        Zadovoljavanje bilo kojeg od uvjeta tj. "grane" rezultira izlaskom iz "stabla" i nastavljanjem izvršavanja daljnjeg koda.
    \end{center}
    \begin{center}
        \footnotesize
        Ni \texttt{elif} ni \texttt{else} sekcije nisu obavezne. \\
        ... \texttt{elif} sekcija nije nužna ako se kod dijeli u samo dvije grane. \\
        ... \texttt{else} sekcija nije nužna u slučajevima tipa \textit{else-do-nothing}. \\
    \end{center}

    \section*{\color{NavyBlue} \texttt{\textbf{while}} petlja}
    \begin{tabular}{|>{\tt}p{9.00cm}|>{}p{15.50cm}|}
        \hline
        while \textit{<condition>}:                         & \texttt{<condition>} je uvjet iteracije                                                               \\ 
        \hspace{5mm}\textit{<code block}>                   & kod koji se izvršava u svakoj iteraciji                                                               \\                             
        \hline
    \end{tabular}
    \begin{center}
        \footnotesize
        \texttt{\textbf{while}} petlja se izvršava kad nije unaprijed poznat broj potrebnih iteracija. 
    \end{center}
    \begin{center}
        \footnotesize
        Petlja se izvršava dok je uvjet petlje zadovoljen. Kako bi završila, unutar same petlje mora doći do izmijene uvijeta. 
    \end{center}

    \section*{\color{NavyBlue} \texttt{\textbf{for}} petlja}
    \begin{tabular}{|>{\tt}p{9.00cm}|>{}p{15.50cm}|}
        \hline
        for \textit{<iterator>} in \textit{<iterable>}:   & \texttt{<iterator>} je jedinični element strukture tj. podatkovne kolekcije \texttt{<iterable>}     \\ 
        \hspace{5mm}\textit{<code block}>                 & kod koji se izvršava u svakoj iteraciji                                                             \\                             
        \hline
    \end{tabular}
    \begin{center}
        \footnotesize
        \texttt{\textbf{for}} petlja se izvršava kad je unaprijed poznat broj potrebnih iteracija. 
    \end{center}
    \begin{center}
        \footnotesize
        U svakoj iteraciji iterator dobiva novu vrijednost. Petlja završava tek kad završe sve iteracije.
    \end{center}

    \section*{}
    \begin{tabular}{|>{\tt}p{2.00cm}|>{\tt}p{12.00cm}|>{\tt}p{10.00cm}|}
        \hline
        Struktura & Iterable & Iterator 
        \\ \hline
        Range & range(3, 20, 2) & 3, 5, 7, 9, 11, 13, 15, 17, 19
        \\ \hline
        String & "apple" & a, p, p, l, e
        \\ \hline
        List & ["apple", "banana", "cherry"] & apple, banana, cherry
        \\ \hline
        Tuple & ("apple", "banana", "cherry") & apple, banana, cherry
        \\ \hline             
        Set & \{"apple", "banana", "cherry"\} & apple, banana, cherry
        \\ \hline
        Dictionary & \{"brand": "Ford", "model": "Mustang", "year": 1964\} & brand, model, year
        \\ \hline
    \end{tabular}

    \section*{\color{NavyBlue} Naredba \texttt{\textbf{break}}}
    \begin{tabular}{|>{\tt}p{9.00cm}|>{}p{15.50cm}|}
        \hline
        for \textit{<iterator>} in \textit{<iterable>}:     &                                                                                                       \\
        \hspace{5mm}if \textit{<condition>}:              & postavljanje uvjeta koji prekida petlju                                                               \\ 
        \hspace{5mm}break                                   & prekid petlje                                                                                         \\ 
        \hspace{5mm}\textit{<code block>}                   & kod koji se (u suprotnom) izvršava u svakoj iteraciji                                                 \\                             
        \hline
    \end{tabular}
    \begin{center}
        \footnotesize
        \texttt{break} služi kako bi se \textbf{prekinulo} izvršavanje \textbf{petlje}.
    \end{center}

    \section*{\color{NavyBlue} Naredba \texttt{\textbf{continue}}}
    \begin{tabular}{|>{\tt}p{9.00cm}|>{}p{15.50cm}|}
        \hline
        for \textit{<iterator>} in \textit{<iterable>}:     &                                                                                                       \\
        \hspace{5mm}if \textit{<condition>}:                & postavljanje uvjeta kojim se preskače trenutna iteracija                                              \\ 
        \hspace{5mm}continue                                & preskok iteracije                                                                                     \\ 
        \hspace{5mm}\textit{<code block>}                   & kod koji se (u suprotnom) izvršava u svakoj iteraciji                                                 \\                             
        \hline
    \end{tabular}
    \begin{center}
        \footnotesize
        \texttt{continue} služi kako bi se \textbf{preskočilo} izvršavanje \textbf{iteracije}.
    \end{center}

    \section*{\color{NavyBlue} Definiranje funkcije}
    \begin{tabular}{|>{\tt}p{9.00cm}|>{}p{15.50cm}|}
        \hline
        def \textit{name}(\textit{params}): & definiranje imena funkcije i postavljanje parametara odvojenih zarezom, koji se koriste u bloku koda
        \\
        \hspace{5mm}\textit{<code block}> & blok koda kojeg će funkcija izvršavati svakim pozivanjem
        \\
        \hspace{5mm}return \textit{value} & povrat (rezultat) funkcije
        \\ \hline
    \end{tabular}
    \begin{center}
        \footnotesize
        \texttt{return} vraća rezultat funkcije i \textbf{izlazi iz funkcije}, slično kao i \texttt{break} \\
        funkcija ne mora imati \texttt{return} ako ne vraća rezultat, npr. ako radi samo \texttt{print} \\
        funkcija ne mora imati \texttt{params} ako nema ulazne podatke. \\
        \texttt{z = f(x,y)} \\
        \texttt{z - \textit{value}} \\
        \texttt{f - \textit{name}} \\
        \texttt{x,y - \textit{params}} \\
    \end{center}

    \section*{\color{NavyBlue} Definiranje funkcije s *args}
    \begin{tabular}{|>{\tt}p{9.00cm}|>{}p{15.50cm}|}
        \hline
        def \textit{name}(\textit{params, *args}): & ... \texttt{*args} sakuplja \textbf{ostatak pozicijskih parametara} u \textbf{n-torku} (nepoznatog broj članova)
        \\
        \hspace{5mm}\textit{<code block}> & blok koda kojeg će funkcija izvršavati svakim pozivanjem
        \\
        \hspace{5mm}return \textit{value} & povrat (rezultat) funkcije
        \\ \hline
    \end{tabular}
    \begin{center}
        \footnotesize
        Budući da \texttt{*args} sakuplja \textbf{ostatak pozicijskih parametara}, nije nužno da se prilikom poziva funkcije u tom parametru nešto nađe, tj. \textbf{nije obavezan}.
    \end{center}

    \section*{\color{NavyBlue} Definiranje funkcije s **kwargs}
    \begin{tabular}{|>{\tt}p{9.00cm}|>{}p{15.50cm}|}
        \hline
        def \textit{name}(\textit{params, **kwargs}): & ... \texttt{**kwargs} sakuplja \textbf{ostatak parametara ključnih riječi} u \textbf{rječnik} (nepoznatog broj članova)
        \\
        \hspace{5mm}\textit{<code block}> & blok koda kojeg će funkcija izvršavati svakim pozivanjem
        \\
        \hspace{5mm}return \textit{value} & povrat (rezultat) funkcije
        \\ \hline
    \end{tabular}
    \begin{center}
        \footnotesize
        Budući da \texttt{*args} sakuplja \textbf{ostatak pozicijskih parametara}, nije nužno da se prilikom poziva funkcije u tom parametru nešto nađe, tj. \textbf{nije obavezan}.
    \end{center}

    \section*{\color{NavyBlue} Redoslijed parametara prilikom definiranja}
    \begin{center}
        \large \ttfamily
        def \textit{name}(\textit{params, *args, default\_params, **kwargs})
    \end{center}

    \section*{\color{NavyBlue} Pozivanje funkcije}
    \begin{tabular}{|>{\tt}p{9.00cm}|>{}p{15.50cm}|}
        \hline
        \textit{name}(\textit{args})                           & pozivanje funkcije koja nije imala \texttt{return}
        \\ \hline
        \textit{variable} = \textit{name}(\textit{args})       & pozivanje funkcije koja je imala \texttt{return} i pohranjivanje njenog rezultata u \texttt{variable}
        \\ \hline
    \end{tabular}
    \begin{center}
        \footnotesize
        \texttt{a = f(2,5)} \\
        \texttt{a - \textit{variable}} \\
        \texttt{f - \textit{name}} \\
        \texttt{2,5 - \textit{args}} \\
    \end{center}
    

    \section*{\color{NavyBlue} Raspakiravanje liste u pojedinačne argumente}
    \begin{tabular}{|>{\tt}p{9.00cm}|>{}p{15.50cm}|}
        \hline
        \textit{name}(\textit{*list}) & članovi liste će se \textbf{raspakirati} u \textbf{pojedinačne argumente} funkcije
        \\ \hline
    \end{tabular}

    \section*{\color{NavyBlue} Parametri funkcije}
    \begin{tabular}{|>{\tt}p{9.00cm}|>{}p{15.50cm}|}
        \hline
        def student(ime, prezime="Horvat", godina=1):                           & definiranje funkcije \texttt{student} s 1 obaveznim i 2 opcionalna parametara \\
        \hspace{5mm}print(ime, prezime, "je", godina, '. godina')               & ispis funkcije \\ \hline
    \end{tabular}
    \begin{center}
        \footnotesize
        Prvo se definiraju svi obavezni parametri, a zatim svi opcionalni parametri.
    \end{center}

    \section*{\color{NavyBlue} Pozicijski argumenti}
    \begin{tabular}{|>{\tt}p{9.00cm}|>{}p{15.50cm}|}
        \hline
        student("Ivan")               & Ivan Horvat je 1. godina \\ \hline
        student("Ivan", "Kovač", 2)   & Ivan Kovač je 2. godina \\ \hline
        student("Ivan", "Kovač")      & Ivan Kovač je 1. godina \\ \hline
        student("Ivan", 2)            & Ivan 2 je 1. godina \\ \hline
    \end{tabular}
    \begin{center}
        \footnotesize
        Pozicijski argumenti zahtjevaju definirani redoslijed i dodjelujuje se s lijevo na desno. \\
        "Nespareni" argumenti dobivaju podrazumjevanu vrijednost.
    \end{center}

    \section*{\color{NavyBlue} Argumenti ključnih riječi}
    \begin{tabular}{|>{\tt}p{9.00cm}|>{}p{15.50cm}|}
        \hline
        student(ime="Ivan")                         & Ivan Horvat je 1. godina \\ \hline
        student(ime="Ivan", godina=2)               & Ivan Horvat je 2. godina \\ \hline
        student(prezime="Kovač", ime="Ivan")        & Ivan Kovač je 1. godina \\ \hline
    \end{tabular}
    \begin{center}
        \footnotesize
        Argumenti ključnih riječi ne zahtjevaju definirani redoslijed. \\
        Nedefinirani argumenti dobivaju podrazumjevanu vrijednost.
    \end{center}

    \section*{\color{NavyBlue} Miješani argumenti}
    \begin{tabular}{|>{\tt}p{9.00cm}|>{}p{15.50cm}|}
        \hline
        student("Ivan", godina=2)                   & Ivan Horvat je 2. godina \\ \hline
        student("Ivan", "Kovač", godina=2)          & Ivan Kovač je 2. godina \\ \hline
    \end{tabular}
    \begin{center}
        \footnotesize
        Pozicijski argumenti se definiraju prije argumenata ključnih riječi. \\
    \end{center}

    \section*{\color{NavyBlue} Primjeri krivog pozivanja funkcija}
    \begin{tabular}{|>{\tt}p{9.00cm}|>{}p{15.50cm}|}
        \hline
        student()                               & pozivanje funkcije bez obaveznih argumenata \\ \hline
        student(ime="Ivan", 2)                  & definiranje pozicijskog argumenta nakon onog s ključnom riječi \\ \hline
        student("Ivan", 2, prezime="Kovač")     & dvostruko definiranje argumenta (pozicija 2 i ključna riječ "prezime") \\ \hline
        student(kolegij="Matematika")           & definiranje nepostojećeg parametra \\ \hline
    \end{tabular}

    \section*{\color{NavyBlue} Vidljivost varijabli}
    \begin{itemize}
        \item \textbf{globalne} varijable definirane su u glavnom tijelu programa i vidljive su svim funkcijama
        \begin{itemize}
            \item varijable definirane unutar neke petlje (npr. iteratori) vidljive su i izvan te petlje
            \item varijable definirane u bloku koda unutar grananja vidljive su i izvan grananja
        \end{itemize}
        \item \textbf{lokalne} varijable definirane su unutar neke funkcije i vidljve su toj funkciji i njenim pod-funkcijama
        \begin{itemize}
            \item varijable definirane unutar funkcije mogu se postaviti globalnima korištenjem ključne riječi \texttt{global}
        \end{itemize}
    \end{itemize}
    \begin{tabular}{|>{\tt}p{9.00cm}|>{}p{15.50cm}|}
        \hline
        global \textit{variable} & postavljanje \texttt{variable} globalnom
        \\
        \textit{variable}=\textit{value} & definiranje vrijednosti varijable
        \\ \hline
    \end{tabular}
    \begin{center}
        \footnotesize
        Ako je određena varijabla definirana globalno, a zatim i više puta lokalno (rekurzivno) unutar funkcija, njena vrijednost u najunutarnjijoj funkciji imat će "najlokalniju" vidljivu vrijednost. \\
    \end{center}

    \section*{\color{NavyBlue} Liste}
    \begin{tabular}{|>{\tt}p{9.00cm}|>{}p{15.50cm}|}
        \hline
        \textit{list} = [1, 2, 3] & definiranje liste
        \\ \hline
        \textit{list} = [[1, 2, 3], [1, 2, 3], [1, 2, 3]] & definiranje ugniježđene liste
        \\ \hline
    \end{tabular}
    \begin{center}
        \footnotesize
        Liste su \textbf{uređene} strukture podataka što znači da je postoji redoslijed članova. \\
        Članovi liste ne moraju biti isti tipovi podataka. \\
        Liste su identiteno-promijenjive... \\
        \texttt{
            >> list1 = [12, 9, 3, 7] \\
            >> list2 = list1 \\
            >> id(list1) == id(list2) \\
            True \\
            >> list1.append(1) \\
            >> list2 \\
            {[12, 9, 3, 7, 1]} \\
            >> id(list1) == id(list2) \\
            True
        } \\
        ... što znači da promjena jedne veže promjenu druge.
    \end{center}

    \section*{\color{NavyBlue} Osnovne operacije s listama}
    \begin{tabular}{|>{\tt}p{9.00cm}|>{}p{15.50cm}|}
        \hline
        \textit{list}[\textit{index}] & indeksiranje lista 
        \\ \hline
        \textit{list}[\textit{[start]}:\textit{[end]}\textit{[:step]}] & komad liste 
        \\ \hline
        \textit{list}[\textit{index}] = \textit{value} & postavljanje nove vrijednosti člana niza
        \\ \hline
        \textit{list}[\textit{[start]}:\textit{[end]}\textit{[:step]}] = \textit{list} & postavljanje nove vrijednosti komada liste s drugom listom (brisanje i umetanje)
        \\ \hline
        [1, 2, 3] + [4, 5, 6] & povezivanje listi (\texttt{[1, 2, 3, 4, 5, 6]})
        \\ \hline
        [1, 2, 3] * 2 & umnožavanje listi (\texttt{[1, 2, 3, 1, 2, 3]})
        \\ \hline
        del \textit{list}[\textit{start}\textit{[:end]}\textit{[:step]}] & briše član ili komad liste
        \\ \hline
        color = [255, 43, 19] & definiranje liste \#...
        \\
        red, green, blue = color & ... i raspakiravanje - pridruživanje po elemantima
        \\ \hline
        item = [4, "Pizza", "Plain", 16.98] & definiranje liste \#\#...
        \\
        quantity, *others, price = item & ... i raspakiravanje - pridruživanje po elemantima
        \\ \hline
    \end{tabular}
    \begin{center}
        \footnotesize
        Posljednji član liste ima indeks -1, predposljednji -2, itd. \\
        \# Lista se može jednostavo rastaviti ako ima jednak broj elemenata. \\
        \#\# Lista se može "složeno" rastaviti tako da jedan element (označen s *) sakuplja sav višak.
    \end{center}

    \section*{\color{NavyBlue} Iteriranje kroz liste}
    \begin{tabular}{|>{\tt}p{9.00cm}|>{}p{15.50cm}|}
        \hline
        for \textit{item} in \textit{list}: & iteriranje kroz listu
        \\ \hline
    \end{tabular}

    \section*{\color{NavyBlue} Metode nad listama}
    \begin{tabular}{|>{\tt}p{9.00cm}|>{\tt}p{0.25cm}|>{}p{14.50cm}|}
        \hline
        \textit{list}.append(\textit{object}) & \ding{55} & dodavanje objekta na kraj \textbf{originalne} liste
        \\ \hline
        \textit{list}.extend(\textit{iterable}) & \ding{55} & dodavanje rastavljene iterable na kraj \textbf{originalne} liste
        \\ \hline
        \textit{list}.insert(\textit{index}, \textit{object}) & \ding{55} & dodavanje objekta ispred člana pod indeksom na \textbf{originalnoj} listi 
        \\ \hline
        \textit{list}.index(\textit{value}) & \ding{51} & vraća prvi indeks na kojem se nalazi \textbf{vrijednost}
        \\ \hline
        \textit{list}.clear() & \ding{55} & prazni \textbf{originalnu} listu
        \\ \hline
        \textit{list}.remove(\textit{value}) & \ding{55} & briše prvi član u \textbf{originalnoj} listi koji ima vrijednost \textit{value}
        \\ \hline
        \textit{list}.pop([\textit{index}]) & \ding{51} & uklanja zadnji član u \textbf{originalnoj} listi (član pod indeksom) i vraća uklonjenu vrijednost
        \\ \hline
        \textit{list}.count(\textit{value}) & \ding{51} & ispisuje broj ponavljanja "value"
        \\ \hline
        \textit{list}.reverse() & \ding{55} & invertira \textbf{originalnu} listu
        \\ \hline
        \textit{list}.sort([reverse=True]) & \ding{55} & (naopako) sortira \textbf{originalnu} listu
        \\ \hline
        \textit{list.copy()} & \ding{51} & kopira listu (korisno jer su liste identiteno-promjenljive)
        \\ \hline
    \end{tabular}
    \begin{center}
        \ding{55} - metoda \textbf{nema} povrat tj. \texttt{return} \\
        \ding{51} - metoda \textbf{ima} povrat tj. \texttt{return} \\
    \end{center}

    \section*{\color{NavyBlue} Rječnici}
    \begin{tabular}{|>{\tt}p{9.00cm}|>{}p{15.50cm}|}
        \hline
        \textit{dict} = \{ & definiranje rječnika
            \\
            \hspace{10pt} \textit{key}: \textit{value},
            \\
            \hspace{10pt} \textit{key}: \textit{value}
            \\
        \}
        \\ \hline
        \textit{dict} = \{ & definiranje ugniježđenog rječnika \\
            \hspace{10pt} \textit{outer\_key}: \{
            \\
            \hspace{10pt} \hspace{10pt} \textit{inner\_key}: \textit{value},
            \\
            \hspace{10pt} \hspace{10pt} \textit{inner\_key}: \textit{value}
            \\
            \hspace{10pt} \},
            \\
            \hspace{10pt} \textit{outer\_key}: \{ \\
            \hspace{10pt} \hspace{10pt} \textit{inner\_key}: \textit{value},
            \\
            \hspace{10pt} \hspace{10pt} \textit{inner\_key}: \textit{value}
            \\
            \hspace{10pt} \}
            \\
            \}
        \\ \hline
    \end{tabular}
    \begin{center}
        \footnotesize
        \textit{key} mora biti \textbf{nepromjenljivi} tip objekta, \textit{value} može biti bilo koji tip objekta. \\
        Ako liste promatramo kao parove indeks-vrijednost, onda rječnike možemo promatrati kao parove ključ-vrijednost. \\
        Drugim riječima, u listama je vrijednost pohranjena na lokaciji indeksa, a u rječnicima na lokaciji ključa. \\
        Iz tog razloga, rječnici služe za grupiranje podataka, ali rječnici nisu \textbf{nisu uređeni} objekti.
        Rječnici su identiteno-promijenjivi... \\
        \texttt{
            >> dict1 = {1: "one"} \\
            >> dict2 = dict1 \\
            >> id(dict1) == id(dict2) \\
            True \\
            >> dict2[2] = "two" \\
            >> dict1 \\
            \{1: "one", 2: "two"\} \\
            >> id(dict1) == id(dict2) \\
            True
        }
        ... što znači da promjena jedne veže promjenu druge.
    \end{center}

    \section*{\color{NavyBlue} Osnovne operacije s rječnicima}
    \begin{tabular}{|>{\tt}p{9.00cm}|>{}p{15.50cm}|}
        \hline
        \textit{dict}[\textit{key}] & "indeksiranje" rječnika, odnosno dohvaćanje \textbf{vrijednosti} ključa
        \\ \hline
        \textit{dict}[\textit{key}] = \textit{value} & postavljanje nove \textbf{vrijednosti} već postojećeg ili novog \textbf{para}
        \\ \hline
        del \textit{dict}[\textit{key}] & briše \textbf{par}
        \\ \hline
        \textit{dict3} = \{**\textit{dict1}, **\textit{dict2}\} & spajanje rječnika 1 i 2 u rječnik 3
        \\ \hline
        \textit{dict3} = \textit{dict1} | \textit{dict2} & spajanje rječnika 1 i 2 u rječnik 3
        \\ \hline
    \end{tabular}
    \begin{center}
        \footnotesize
        Ključevi moraju biti jedinstveni. \\
        Prilikom manipulacije rječnika, mijenjaju se njihove \textbf{vrijednosti}, a ne \textbf{ključevi}.
    \end{center}

    \section*{\color{NavyBlue} Metode nad rječnicima}
    \begin{tabular}{|>{\tt}p{9.00cm}|>{\tt}p{0.25cm}|>{}p{14.50cm}|}
        \hline
        \textit{dict}.get(\textit{key}) & \ding{51} & vraća vrijednost ključa ako taj ključ postoji, a u suprotnom vraća \texttt{None}
        \\ \hline
        \textit{dict}.pop(\textit{key}) & \ding{51} & uklanja \textbf{par} ključ-vrijednost u \textbf{originalnom} rječniku i vraća uklonjenu \textbf{vrijednost}
        \\ \hline
        \textit{dict}.popitem() & \ding{51} & uklanja zadnje dodan \textbf{par} u \textbf{originalnom} rječniku i vraća uklonjen \textbf{par} kao \textbf{tuple}
        \\ \hline
        \textit{dict}.clear() & \ding{55} & prazni \textbf{originalni} rječnik
        \\ \hline
        \textit{dict}.keys() & \ding{51} & vraća "listu" \textbf{ključeva} (objekt tipa \texttt{dict\_keys})
        \\ \hline
        \textit{dict}.values() & \ding{51} & vraća "listu" \textbf{vrijednosti} (objekt tipa \texttt{dict\_values})
        \\ \hline
        \textit{dict}.items() & \ding{51} & vraća "listu tupleova", tj. "listu" \textbf{parova} (objekt tipa \texttt{dict\_items})
        \\ \hline
        \textit{dict}.update(\textit{dict}) & \ding{55} & osvježavanje \textbf{originalnog} rječnika s parovima drugog rječnika
        \\ \hline
    \end{tabular}
    \begin{center}
        \ding{55} - metoda \textbf{nema} povrat tj. \texttt{return} \\
        \ding{51} - metoda \textbf{ima} povrat tj. \texttt{return} \\
    \end{center}

    \section*{\color{NavyBlue} Iteriranje kroz rječnike}
    \begin{tabular}{|>{\tt}p{9.00cm}|>{}p{15.50cm}|}
        \hline
        for \textit{key} in \textit{dict}: & iteriranje kroz rječnik po \textbf{ključevima}
        \\ \hline
        for \textit{key} in \textit{dict}.keys(): & iteriranje kroz rječnik po \textbf{ključevima}
        \\ \hline
        for \textit{value} in \textit{dict}.values(): & iteriranje kroz rječnik po \textbf{vrijednostima}
        \\ \hline
        for \textit{key}, \textit{value} in \textit{dict}.items(): & iteriranje kroz rječnik po \textbf{parovima}
        \\ \hline
    \end{tabular}

    \section*{\color{NavyBlue} N-terci}
    \begin{tabular}{|>{\tt}p{9.00cm}|>{}p{15.50cm}|}
        \hline
        \textit{tuple} = (1, 2, 3,) & definiranje n-terca
        \\ \hline
        \textit{tuple} = ((1, 2, 3),(1, 2, 3),(1, 2, 3),) & definiranje ugniježđenog n-terca
        \\ \hline
    \end{tabular}
    \begin{center}
        \footnotesize
        N-terci su \textbf{uređene} strukture podataka što znači da je postoji redoslijed članova. \\
        Za razliku od listi, kad se jednom kreiraju, ne mogu se mijenjati. \\
        Preporuka je koristiti zarez na kraju zadnjeg elemeneta. Ako postoji samo jedan element, zarez je \textbf{obavezan}. \\
        Elementi n-terca mogu biti bilo koji tipovi objekta. \\
    \end{center}

    \section*{\color{NavyBlue} Osnovne operacije s n-tercima}
    \begin{tabular}{|>{\tt}p{9.00cm}|>{}p{15.50cm}|}
        \hline
        \textit{tuple}[\textit{index}] & indeksiranje n-terca 
        \\ \hline
        \textit{tuple}[\textit{[start]}:\textit{[end]}\textit{[:step]}] & komad liste 
        \\ \hline
    \end{tabular}

    \section*{\color{NavyBlue} Metode nad n-tercima}
    \begin{tabular}{|>{\tt}p{9.00cm}|>{\tt}p{0.25cm}|>{}p{14.50cm}|}
        \hline
        \textit{tuple}.index(\textit{value}) & \ding{51} & vraća prvi indeks na kojem se nalazi \textbf{vrijednost}
        \\ \hline
        \textit{tuple}.count(\textit{value}) & \ding{51} & ispisuje broj ponavljanja \textbf{vrijednosti}
        \\ \hline
    \end{tabular}
    \begin{center}
        \ding{55} - metoda \textbf{nema} povrat tj. \texttt{return} \\
        \ding{51} - metoda \textbf{ima} povrat tj. \texttt{return} \\
    \end{center}

    \section*{\color{NavyBlue} Skupovi}
    \begin{tabular}{|>{\tt}p{9.00cm}|>{}p{15.50cm}|}
        \hline
        \textit{set} = \{1, 2, 3\} & definiranje skupa
        \\ \hline
        \textit{set} = set() & definiranje praznog skupa (jer je \texttt{\{\}} zauzeto za definiranje rječnika)
        \\ \hline
    \end{tabular}
    \begin{center}
        \footnotesize
        Svi elementi skupa moraju biti \textbf{nepromjenljivi} tipovi objekta. \\
        Set je kao rječnik, no ključevi nemaju par. \\
        Setovi se ne mogu indeksirati jer su elementi "nasumično poslagani". \\
        Set se zbog \textbf{sintakse definiranja}, \textbf{neindeksiranja} te \textbf{jedinstvenosti} članova može promatrati kao \textbf{niz ključeva}.
    \end{center}

    \section*{\color{NavyBlue} Osnovne operacije sa skupovima}
    \begin{tabular}{|>{\tt}p{9.00cm}|>{}p{15.50cm}|}
        \hline
        \textit{set} = set(\textit{list}) & pretvaranje liste u skup kako bi se uklonili duplikati
        \\ \hline
    \end{tabular}

    \section*{\color{NavyBlue} Metode nad skupovima}
    \begin{tabular}{|>{\tt}p{9.00cm}|>{\tt}p{0.25cm}|>{}p{14.50cm}|}
        \hline
        \textit{set}.add(\textit{value}) & \ding{55} & dodaje \textbf{vrijednost} u \textbf{originalni} skup
        \\ \hline
        \textit{set}.remove(\textit{value}) & \ding{55} & uklanja \textbf{vrijednost} u \textbf{originalnom} skupu i generira grešku ako je nema
        \\ \hline
        \textit{set}.discard(\textit{value}) & \ding{55} & uklanja \textbf{vrijednost} u \textbf{originalnom} skupu, ali ne generira grešku ako je nema
        \\ \hline
        \textit{set}.clear() & \ding{55} & briše sadržaj liste
        \\ \hline
        \textit{set}.len() & \ding{55} & vraća broj elemenata skupa
        \\ \hline
    \end{tabular}
    \begin{center}
        \ding{55} - metoda \textbf{nema} povrat tj. \texttt{return} \\
        \ding{51} - metoda \textbf{ima} povrat tj. \texttt{return} \\
    \end{center}

    \section*{\color{NavyBlue} Operacija sa skupovima}
    \begin{tabular}{|>{\tt}p{9.00cm}|>{}p{15.50cm}|}
        \hline
        \textit{set1} \& \textit{set2} & \textbf{presjek}, \texttt{set1} $\cap$ \texttt{set2} (zajedničke vrijednosti skupova)
        \\ \hline
        \textit{set1} | \textit{set2} & \textbf{unija}, \texttt{set1} $\cup$ \texttt{set2} (sve vrijednosti skupova)
        \\ \hline
        \textit{set1} - \textit{set2} & \textbf{skupovna razlika}, \texttt{set1} $\setminus$ \texttt{set2} (\texttt{set1} umanjen za sve elemente \texttt{set2}, tj. jedinstveni elementi \texttt{set1})
        \\ \hline
        \textit{set2} - \textit{set1} & \textbf{skupovna razlika}, \texttt{set2} $\setminus$ \texttt{set1} (\texttt{set2} umanjen za sve elemente \texttt{set1}, tj. jedinstveni elementi \texttt{set2})
        \\ \hline
    \end{tabular}

    \section*{\color{NavyBlue} Tipovi grešaka}
    \begin{tabular}{|>{\tt}p{9.00cm}|>{}p{15.50cm}|}
        \hline
        SyntaxError & korištenje zabranjenih znakova (kao što je @), indentacijske greške, nepostojeće ":" nakon petlji, grana
        \\ \hline
        NameError & korištenje nepostojećih naredbi, ključnih riječi ili varijabli
        \\ \hline
        IndexError & pokušaj pristupanja nepostojećem indeksu liste ili n-torke
        \\ \hline
        KeyError & pokušaj pristupanja nepostojećeg ključa u rječniku
        \\ \hline
        TypeError & pokušaj manipulacije pogrešnog tipa objekata, npr. zbrajanja integera i stringa
        \\ \hline
        ValueError & korištenje pogrešne vrijednosti (ali dobrog tipa) unutar funkcije
        \\ \hline
    \end{tabular}

    \section*{\color{NavyBlue} Greške}
    \begin{tabular}{|>{\tt}p{9.00cm}|>{}p{15.50cm}|}
        \hline
        Raise SyntaxError("\textit{message}") & podiže grešku tipa \texttt{SyntaxError} i ispisuje poruku
        \\ \hline
        Raise NameError("\textit{message}") & podiže grešku tipa \texttt{NameError} i ispisuje poruku
        \\ \hline
        Raise IndexError("\textit{message}") & podiže grešku tipa \texttt{IndexError} i ispisuje poruku
        \\ \hline
        Raise KeyError("\textit{message}") & podiže grešku tipa \texttt{KeyError} i ispisuje poruku
        \\ \hline
        Raise TypeError("\textit{message}") & podiže grešku tipa \texttt{TypeError} i ispisuje poruku
        \\ \hline
        Raise ValueError("\textit{message}") & podiže grešku tipa \texttt{ValueError} i ispisuje poruku
        \\ \hline
    \end{tabular}
    \begin{center}
        \texttt{Raise} se ne koristi zbog korisnika program već zbog ostalih koji na programu rade. \\
        Koristi se kako bi se prekinulo daljnje izvršavanje koda.
    \end{center}

    \section*{\color{NavyBlue} try i except}
    \begin{tabular}{|>{\tt}p{9.00cm}|>{}p{15.50cm}|}
        \hline
        try: & kod koji bi potencijalno mogao generirati grešku 
        \\
        \hspace{5mm} {<code block>} & 
        \\
        except [ErrorType]: & kod koji se izvršava ako se u gornjem bloku generira greška [tipa ErrorType]
        \\
        \hspace{5mm} {<code block>} & 
        \\ \hline
        \hline
        try: &
        \\
        \hspace{5mm} num = int(input("Unesite broj: ")) & kod koji će generirati grešku ako korisnik unese string
        \\
        except ValueError: &
        \\
        \hspace{5mm} num = 1 & broj koji se odabire ako je korisnik unio string
        \\
        \hspace{5mm} print("Pogrešan unos. Odabran je 1.")
        \\
        except EOFError: &
        \\
        \hspace{5mm} num = 1 & broj koji se odabire ako je korisnik izašao iz programa
        \\
        \hspace{5mm} print("Izlazak iz programa. Odabran je 1.")
        \\ \hline
    \end{tabular}
    \begin{center}
        Korištenjem \texttt{try} i \texttt{except} blokova, izbjegnuto je prekidanje izvršavanja koda. 
    \end{center}

    \section*{\color{NavyBlue} Pristupi programiranju}
    \begin{tabular}{|>{\tt}p{9.00cm}|>{}p{15.50cm}|}
        \hline
        year = input("Enter a year: ") & \texttt{Look Before You Leap (LBYL)} \\
        if year.isnumeric(): & \\
        \hspace{5mm} year = int(year) & \\
        else: & \\
        \hspace{5mm} year = 2025 & \\
        \hline
        try: & \texttt{Easier to Ask Forgiveness than Permission (EAFP)} \\
        \hspace{5mm} year = int(input("Enter a year: ")) & \\
        except ValueError: & \\
        \hspace{5mm} year = 2025 & \\
        \hline
    \end{tabular}
    \begin{center}
        \texttt{EAFP} se smatra "više Pythonski".
    \end{center}

    \section*{\color{NavyBlue} Uvoz cijelih modula}
    \begin{tabular}{|>{\tt}p{6.00cm}|>{\tt}p{6.00cm}|>{}p{12.00cm}|}
        \hline
        import random & import random as rand & uvoz \textbf{modula} \texttt{random} (pod nazivom \texttt{rand})
        \\
        random.randint(1, 100) & rand.randint(1, 100) & korištenje \textbf{metode} \texttt{randint} \textbf{modula} \texttt{random}
        \\ \hline
        import calendar & import calendar as cal & uvoz \textbf{modula} \texttt{calendar} (pod nazivom \texttt{cal})
        \\
        calendar.isleap(2023) & cal.isleap(2023) & korištenje \textbf{metode} \texttt{isleap} \textbf{modula} \texttt{calendar}
        \\ \hline
        import math & import math as m & uvoz \textbf{modula} \texttt{math} (pod nazivom \texttt{m})
        \\
        math.sqrt(2) & m.sqrt(2) & korištenje \textbf{metode} \texttt{sqrt} \textbf{modula} \texttt{math}
        \\ \hline
    \end{tabular}
    \begin{center}
        \textbf{Moduli} su Python skripte koje \textbf{importanjem} donose nove funkcionalnosti.
        \\
        \textbf{Import} se može shvatiti kao da se cijela skripta zalijepi u header.
        \\
        Nakon uvoza, dostupne su sve metode, funkcije i varijable te skripte.
        \\
        Moduli su tipa \texttt{class 'module'}
    \end{center}

    \section*{\color{NavyBlue} Uvoz specifične metode modula}
    \begin{tabular}{|>{\tt}p{8.00cm}|>{\tt}p{8.00cm}|>{}p{8.00cm}|}
        \hline
        from random import randint & from random import randint as ri & uvoz \textbf{metode} \texttt{randint} (pod nazivom \texttt{ri})
        \\
        randint(1, 100) & ri(1, 100) & korištenje \textbf{metode} \texttt{randint}
        \\ \hline
        from calendar import isleap & from calendar import isleap as il & uvoz \textbf{metode} \texttt{isleap} (pod nazivom \texttt{il})
        \\
        isleap(2023) & il(2023) & korištenje \textbf{metode} \texttt{isleap}
        \\ \hline
        from math import sqrt & from math import sqrt as sq & uvoz \textbf{metode} \texttt{sqrt} (pod nazivom \texttt{sq})
        \\
        sqrt(2) & sq(2) & korištenje \textbf{metode} \texttt{sqrt}
        \\ \hline
    \end{tabular}
    \begin{center}
        Argument od \texttt{from} je \textbf{modul}, odnosno \textbf{Python skripta}.
        \\
        Argument od \texttt{import} je \textbf{funkcionalnost}, odnosno \textbf{metoda, funkcija ili varijabla} iz te Python skripte.
        \\
        Moguće je uvesti i \textbf{više} metoda nekog modula tako se \textbf{metode} \textbf{odvoje zarezima}.
        \\
        Moduće je uvesti i \textbf{sve} metode nekog modula tako da se \textbf{zadaje argument \texttt{*}}. To je korisno jer zatim nije potrebno koristiti ime modula kao prefiks.
    \end{center}

    \section*{\color{NavyBlue} Uvoz druge skripte}
    \begin{tabular}{|>{\tt}p{8.00cm}|>{\tt}p{8.00cm}|>{}p{8.00cm}|}
        \hline
        import script & from script import func & uvoz \textbf{skripte} \texttt{script.py}
        \\
        script.func() & func() & korištenje \textbf{funkcije} \texttt{func} definirane u \texttt{script.py}
        \\
        script.var & var & korištenje \textbf{varijable} \texttt{var} definirane u \texttt{script.py}
        \\ \hline
    \end{tabular}
    \begin{center}
        Skripte koja se uvozi mora se nalaziti \textbf{u istom direktoriju} kao i skripta u koju se uvozi.
    \end{center}

    \section*{\color{NavyBlue} \texttt{pip} modul}
    \begin{tabular}{|>{\tt}p{8.00cm}|>{}p{16.50cm}|}
        \hline
        sudo apt install -y python3-pip  & instalacija \texttt{pip} modula
        \\ \hline
        python3 -m pip ---version  & provjera verzije \texttt{pip} modula (za Python 3)
        \\ \hline
        python3 -m pip install \textit{package} & instalacija paketa pomoću \texttt{pip} modula (za Python 3) s \href{https://pypi.org/}{\textbf{pypi.org}}
        \\ \hline
    \end{tabular}
    \begin{center}
        Nakon instalacije, paket se može standardno uvoziti u Python s \texttt{import \textit{package}}.
    \end{center}

\end{document}