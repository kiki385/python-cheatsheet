\documentclass[10pt]{article}

% Paketi koji će se koristiti
\usepackage[a4paper, landscape, margin=2.00cm]{geometry}
\usepackage{amsfonts, amssymb, amsmath}
\usepackage{graphicx}
\usepackage{float}
\usepackage{multirow}
\usepackage{array}
\usepackage[utf8]{inputenc}
\usepackage[T1]{fontenc}
\usepackage{fancyhdr}
\usepackage{lastpage}
\usepackage[dvipsnames]{xcolor}
\usepackage{pifont}

% Dodaj header i footer
\pagestyle{fancy}
\renewcommand{\headrulewidth}{0pt}
\fancyhf{}
\setlength{\headheight}{32.25pt}
\lhead{}
\rhead{\includegraphics[height=1.00cm]{./python.png}}
\setlength{\footskip}{12.00pt}
\lfoot{}
\rfoot{str. {\thepage}/{\pageref{LastPage}}}

\begin{document}

\renewcommand{\arraystretch}{1.50}

    \section*{\color{NavyBlue} Komentiranje}
    \begin{tabular}{|>{\tt}p{9.00cm}|>{}p{15.50cm}|}
        \hline
        \# & početak linijskog komentara \\ \hline
    \end{tabular}

    \section*{\color{NavyBlue} Osnovni tipovi varijabli / objekata}
    \begin{tabular}{|>{\tt}p{9.00cm}|>{}p{15.50cm}|}
        \hline
        str & (string), (uređeni) niz znakova
        \\ \hline
        int & (integer), cijeli broj
        \\ \hline
        float & broj s pomičnim zarezom
        \\ \hline
        bool & (boolean), logička varijabla \textbf{True} ili \textbf{False}
        \\ \hline
    \end{tabular}

    \section*{\color{NavyBlue} Složeni tipovi varijabli / objekata}
    \begin{tabular}{|>{\tt}p{9.00cm}|>{}p{15.50cm}|}
        \hline
        range   & raspon - uređen (indeksiran) niz integera                                             \\ \hline
        list    & lista - uređena (indeksirana) podatkovna kolekcija (niz objekata)                       \\ \hline
    \end{tabular}

    \section*{\color{NavyBlue} Konverzije varijabli}
    \begin{tabular}{|>{\tt}p{9.00cm}|>{}p{15.50cm}|}
        \hline
        str(object)     & konverzija u string  \\ \hline
        int(object)     & konverzija u integer \\ \hline
        float(object)   & konverzija u float   \\ \hline
        bool(object)    & konverzija u boolean \\ \hline
        list(iterable)  & konverzija (podatke kolekcije) u listu \\ \hline
    \end{tabular}

    \section*{\color{NavyBlue} Funkcije s objektima}
    \begin{tabular}{|>{\tt}p{9.00cm}|>{}p{15.50cm}|}
        \hline
        len(\textit{object}) & vraća broj znakova u objektu
        \\ \hline
        print(\textit{object}) & ispisuje objekt (ne vraća ništa)
        \\ \hline
        type(\textit{object}) & ispisuje tip objekta
        \\ \hline
        id(\textit{object}) & ispisuje identitet objekta (adresu u memoriji) *
        \\ \hline
    \end{tabular}
    \begin{center}
        \footnotesize
        * Svi objekti u Pythonu imaju svoj jedinstveni ID, koji se dodjeljuje objektu kada se objekt \textbf{kreira}. \\
        ID objekta je adresa u memoriji i bit će drugačija svaki put kad se pokrene program. \\
        Objekti kojima se vrijednosti mogu mijenjati bez promjene identiteta zovu se promjenljivi (engl. mutable) objekti, a oni kojima se vrijednost ne može mijenjati bez stvaranja novog objekta istog tipa zovu se nepromjenljivi (engl. immutable) objekti. \\
        Promjena vrijednosti objekta obično se događa operatorom pridružbe ili djelovanjem metode na vrijednost objekta. \\
    \end{center}

    \section*{\color{NavyBlue} Ostale funkcije}
    \begin{tabular}{|>{\tt}p{9.00cm}|>{}p{15.50cm}|}
        \hline
        range([start], end, [step]) & vraća sekvencu (raspon) brojeva od "start" do "end", s razlikom od "step"
        \\ \hline
    \end{tabular}

    \section*{\color{NavyBlue} Deklaracija stringova}
    \begin{tabular}{|>{\tt}p{9.00cm}|>{}p{15.50cm}|}
        \hline
        "text"                                  & klasičan način zapisa stringa                                            \\ \hline
        """text"""                              & zapis stringa koji omogućuje prijelaz u novi red                         \\ \hline
        f"text \{\textit{value}\} text"         & formatirani string koji evaluira vrijednost u vitičastim zagradama       \\ \hline
    \end{tabular}
    \begin{center}
        \footnotesize
        \textbf{Prilikom deklaracije, mogu se koristiti i jednostruki navodnici!}
        Stringovi nisu identiteno-promijenjivi... \\
        \texttt{>> string1 = "Ivan"} \\
        \texttt{>> string2 = string1} \\
        \texttt{>> id(string1) == id(string2)} \\
        \texttt{"True"} \\
        \texttt{>> string1 = "Marko"} \\
        \texttt{>> string2} \\
        \texttt{"Ivan"} \\
        \texttt{>> id(string1) == id(string2)} \\
        \texttt{"False"} \\
        ... što znači da promjena jedne ne veže promjenu druge.
    \end{center}
    
    \section*{\color{NavyBlue} Osnovne operacije sa stringovima}
    \begin{tabular}{|>{\tt}p{9.00cm}|>{}p{15.50cm}|}
        \hline
        "text"[0] & indeksiranje stringova (\texttt{"t"})
        \\ \hline
        "text"[0:2:1] & komad stringa (počevši od 0 do 2 (ne uključujući 2), s korakom 1) (\texttt{"te"})
        \\ \hline
        "abc" + "def" & povezivanje stringova (\texttt{"abcdef"})
        \\ \hline
        "abc" * 2 & umnožavanje stringova (\texttt{"abcabc"})
        \\ \hline
    \end{tabular}
    \begin{center}
        \footnotesize
        Posljednji član stringa ima indeks -1, predzadnji -2, itd.
    \end{center}
    
    \section*{\color{NavyBlue} Funkcije sa stringovima}
    \begin{tabular}{|>{\tt}p{9.00cm}|>{}p{15.50cm}|}
        \hline
        input(\textit{prompt}) & traži vrijednost inputa u prompt (tipa string)
        \\ \hline
    \end{tabular}

    \section*{\color{NavyBlue} Metode nad stringovima}
    \begin{tabular}{|>{\tt}p{9.00cm}|>{\tt}p{0.25cm}|>{}p{14.50cm}|}
        \hline
        \textit{string}.isnumeric() & \ding{51} & ispisuje \textbf{True} ako je string broj, a \textbf{False} ako nije
        \\ \hline
        \textit{string}.find(\textit{value}, \textit{[start]}, \textit{[end]}) & \ding{51} & vraća indeks prvog pojavljivanja "value"
        \\ \hline
        \textit{string}.index(\textit{value}, \textit{[start]}, \textit{[end]}) & \ding{51} & slično kao find, ali u slučaju nepronalaska izbacuje grešku
        \\ \hline
        \textit{string}.rfind(\textit{value}, \textit{[start]}, \textit{[end]}) & \ding{51} & vraća indeks zadnjeg pojavljivanja "value"
        \\ \hline
        \textit{string}.rindex(\textit{value}, \textit{[start]}, \textit{[end]}) & \ding{51} & slično kao rfind, ali u slučaju nepronalaska izbacuje grešku
        \\ \hline
        \textit{string}.lstrip(\textit{[character]}) & \ding{51} & uklanja niz "character" (ili razmak) s lijeve strane \textbf{kopije} stringa
        \\ \hline
        \textit{string}.rstrip(\textit{[character]}) & \ding{51} & uklanja niz "character" (ili razmak) s desne strane \textbf{kopije} stringa
        \\ \hline
        \textit{string}.strip(\textit{[character]}) & \ding{51} & uklanja niz "character" (ili razmak) s lijeve i desne strane \textbf{kopije} stringa
        \\ \hline
        \textit{string}.replace(\textit{oldvalue}, \textit{newvalue}, \textit{[count]}) & \ding{51} & mijenja "oldvalue" za "newvalue" (u prvih "count" pojavljivanja) \textbf{kopije} stringa
        \\ \hline
        \textit{string}.count(\textit{value}, \textit{[start]}, \textit{[end]}) & \ding{51} & ispisuje broj ponavljanja "value"
        \\ \hline
        \textit{string}.lower() & \ding{51} & ispisuje \textbf{kopiju} stringa kojemu su sva slova mala
        \\ \hline
        \textit{string}.upper() & \ding{51} & ispisuje \textbf{kopiju} stringa kojemu su sva slova velika
        \\ \hline
        \textit{string}.capitalize() & \ding{51} & ispisuje \textbf{kopiju} stringa kojemu je prvo slovo veliko
        \\ \hline
        \textit{string}.split(\textit{delimiter}) & \ding{51} & ispisuje \textbf{novu listu} nastalu razdvajanjem \textbf{kopije stringa} po graničniku
        \\ \hline
        \textit{delimiter}.join(\textit{iterable}) & \ding{51} & ispisuje \textbf{novi string} nastao spajanjem elemenata \textbf{kopije iterabla} po graničniku
        \\ \hline
    \end{tabular}
    \begin{center}
        \ding{55} - metoda \textbf{nema} povrat tj. \texttt{return}
        \\
        \ding{51} - metoda \textbf{ima} povrat tj. \texttt{return}
        \\
    \end{center}

    \section*{\color{NavyBlue} Aritmetički operatori}
    \begin{tabular}{|>{\tt}p{9.00cm}|>{}p{15.50cm}|}
        \hline
        +  & zbrajanje                      \\ \hline
        -  & oduzimanje                     \\ \hline
        *  & množenje                       \\ \hline
        ** & potenciranje                   \\ \hline
        /  & dijeljenje                     \\ \hline
        // & cjelobrojno dijeljenje         \\ \hline
        \% & ostatak cjelobrojnog djeljenja \\ \hline
    \end{tabular}

    \section*{\color{NavyBlue} Operatori dodjele}
    \begin{tabular}{|>{\tt}p{9.00cm}|>{}p{15.50cm}|}
        \hline
        =   & a = 5             \\ \hline
        +=  & a = a + 5         \\ \hline
        -=  & a = a - 5         \\ \hline
        *=  & a = a * 5         \\ \hline
        **= & a = a ** 5        \\ \hline
        /=  & a = a / 5         \\ \hline
        //= & a = a // 5        \\ \hline
        \%= & a = a \% 5        \\ \hline
    \end{tabular}

    \section*{\color{NavyBlue} Operatori usporedbe vrijednosti}
    \begin{tabular}{|>{\tt}p{9.00cm}|>{}p{15.50cm}|}
        \hline
        ==      & jednako           \\ \hline
        !=      & nije jednako      \\ \hline
        >       & veće              \\ \hline
        >=      & veće ili jednako  \\ \hline
        <       & manje             \\ \hline
        <=      & manje ili jednako \\ \hline
    \end{tabular}
    \begin{center}
        \footnotesize
        Izlaz može biti \texttt{True} ili \texttt{False}
    \end{center}

    \section*{\color{NavyBlue} Operatori usporedbe adresa u memoriji}
    \begin{tabular}{|>{\tt}p{9.00cm}|>{}p{15.50cm}|}
        \hline
        is      & jednakost adresa u memoriji \\ \hline
        is not  & nejednakost adresa u memoriji \\ \hline
    \end{tabular}
    \begin{center}
        \footnotesize
        Operator provjerava \texttt{id(object1) == id(object2)} \\
        Izlaz može biti \texttt{True} ili \texttt{False}
    \end{center}

    \section*{\color{NavyBlue} Logički operatori}
    \begin{tabular}{|>{\tt}p{9.00cm}|>{}p{15.50cm}|}
        \hline
        and   & istinito ako su sve tvrdnje točne           \\ \hline
        or    & istinito ako je barem jedna tvrdnja točna   \\ \hline
        not   & inverzija (negacija) istinitosti tvrdnje    \\ \hline
    \end{tabular}
    \begin{center}
        \footnotesize
        Izlaz može biti \texttt{True} ili \texttt{False}
    \end{center}

    \section*{\color{NavyBlue} Operatori članstva}
    \begin{tabular}{|>{\tt}p{9.00cm}|>{}p{15.50cm}|}
        \hline
        in        & točnost postojanja člana u sekvenci         \\ \hline
        not in    & netočnost postojanja člana u sekvenci       \\ \hline
    \end{tabular}
    \begin{center}
        \footnotesize
        Izlaz može biti \texttt{True} ili \texttt{False}
    \end{center}

    \section*{\color{NavyBlue} Neistinite (lažne) vrijednosti}
    \begin{tabular}{|>{\tt}p{9.00cm}|>{}p{15.50cm}|}
        \hline
        False & definicijska "laž"                                      \\ \hline
        None  & "vrijednost" varijable bez vrijednosti (a = None)       \\ \hline                            
        0     & 0 tipa cijelog broja                                    \\ \hline
        0.0   & 0 tipa broja s pomičnim zarezom                         \\ \hline
        ""    & prazni string (u bilo kojem formatu)                    \\ \hline
        []    & prazan niz                                              \\ \hline
        ()    & prazna n-torka                                          \\ \hline
        \{\}  & prazan rječnik                                          \\ \hline
        set() & prazan skup                                             \\ \hline
        range(0) & prazan raspon                                        \\ \hline
    \end{tabular}

    \section*{\color{NavyBlue} Grananje}
    \begin{tabular}{|>{\tt}p{9.00cm}|>{}p{15.50cm}|}
        \hline
        if \textit{<condition 1>}:                   & postavljanje prvog uvjeta                                         \\ 
        \hspace{5mm}<code block 1>                   & kod koji se izvršava ako je prvi uvjet zadovoljen                 \\                             
        elif \textit{<condition 2>}:                 & postavljanje drugog uvjeta                                        \\ 
        \hspace{5mm}\textit{<code block 2}>          & kod koji se izvršava ako je drugi uvjet zadovoljen                \\ 
        else:                                        & pokrivanje svih ostalih uvjeta                                    \\ 
        \hspace{5mm}\textit{<code block 3}>          & kod koji se izvršava ako prvi i drugi uvjet nisu zadovoljeni      \\ \hline
    \end{tabular}
    \begin{center}
        \footnotesize
        \texttt{if}, \texttt{elif} i \texttt{else} moraju koristiti iste indentacije! \\
        \texttt{<code block 1>}, \texttt{<code block 2>} i \texttt{<code block 3>} moraju koristiti iste indentacije!
    \end{center}
    \begin{center}
        \footnotesize
        Zadovoljavanje bilo kojeg od uvjeta tj. "grane" rezultira izlaskom iz "stabla" i nastavljanjem izvršavanja daljnjeg koda.
    \end{center}
    \begin{center}
        \footnotesize
        Ni \texttt{elif} ni \texttt{else} sekcije nisu obavezne. \\
        ... \texttt{elif} sekcija nije nužna ako se kod dijeli u samo dvije grane. \\
        ... \texttt{else} sekcija nije nužna u slučajevima tipa \textit{else-do-nothing}. \\
    \end{center}

    \section*{\color{NavyBlue} \texttt{\textbf{while}} petlja}
    \begin{tabular}{|>{\tt}p{9.00cm}|>{}p{15.50cm}|}
        \hline
        while \textit{<condition>}:                         & \texttt{<condition>} je uvjet iteracije                                                               \\ 
        \hspace{5mm}\textit{<code block}>                   & kod koji se izvršava u svakoj iteraciji                                                               \\                             
        \hline
    \end{tabular}
    \begin{center}
        \footnotesize
        \texttt{\textbf{while}} petlja se izvršava kad nije unaprijed poznat broj potrebnih iteracija. 
    \end{center}
    \begin{center}
        \footnotesize
        Petlja se izvršava dok je uvjet petlje zadovoljen. Kako bi završila, unutar same petlje mora doći do izmijene uvijeta. 
    \end{center}

    \section*{\color{NavyBlue} \texttt{\textbf{for}} petlja}
    \begin{tabular}{|>{\tt}p{9.00cm}|>{}p{15.50cm}|}
        \hline
        for \textit{<iterator>} in \textit{<iterable>}:   & \texttt{<iterator>} je jedinični element strukture tj. podatkovne kolekcije \texttt{<iterable>}     \\ 
        \hspace{5mm}\textit{<code block}>                 & kod koji se izvršava u svakoj iteraciji                                                             \\                             
        \hline
    \end{tabular}
    \begin{center}
        \footnotesize
        \texttt{\textbf{for}} petlja se izvršava kad je unaprijed poznat broj potrebnih iteracija. 
    \end{center}
    \begin{center}
        \footnotesize
        U svakoj iteraciji iterator dobiva novu vrijednost. Petlja završava tek kad završe sve iteracije.
    \end{center}

    \section*{}
    \begin{tabular}{|>{\tt}p{2.00cm}|>{\tt}p{12.00cm}|>{\tt}p{10.50cm}|}
        \hline
        Struktura   & Iterable                                                      & Iterator    \\ \hline
        Range       & range(3, 20, 2)                                               & 3, 5, 7, 9, 11, 13, 15, 17, 19    \\ \hline
        String      & "apple"                                                       & a, p, p, l, e                     \\ \hline
        List        & ["apple", "banana", "cherry"]                                 & apple, banana, cherry             \\ \hline
        Tuple       & ("apple", "banana", "cherry")                                 & apple, banana, cherry             \\ \hline             
        Set         & \{"apple", "banana", "cherry"\}                               & apple, banana, cherry             \\ \hline
        Dictionary  & \{"brand": "Ford", "model": "Mustang", "year": 1964\}         & brand, model, year                \\ \hline
    \end{tabular}

    \section*{\color{NavyBlue} Naredba \texttt{\textbf{break}}}
    \begin{tabular}{|>{\tt}p{9.00cm}|>{}p{15.50cm}|}
        \hline
        for \textit{<iterator>} in \textit{<iterable>}:     &                                                                                                       \\
        \hspace{5mm}if \textit{<condition>}:              & postavljanje uvjeta koji prekida petlju                                                               \\ 
        \hspace{5mm}break                                   & prekid petlje                                                                                         \\ 
        \hspace{5mm}\textit{<code block>}                   & kod koji se (u suprotnom) izvršava u svakoj iteraciji                                                 \\                             
        \hline
    \end{tabular}
    \begin{center}
        \footnotesize
        \texttt{break} služi kako bi se \textbf{prekinulo} izvršavanje \textbf{petlje}.
    \end{center}

    \section*{\color{NavyBlue} Naredba \texttt{\textbf{continue}}}
    \begin{tabular}{|>{\tt}p{9.00cm}|>{}p{15.50cm}|}
        \hline
        for \textit{<iterator>} in \textit{<iterable>}:     &                                                                                                       \\
        \hspace{5mm}if \textit{<condition>}:                & postavljanje uvjeta kojim se preskače trenutna iteracija                                              \\ 
        \hspace{5mm}continue                                & preskok iteracije                                                                                     \\ 
        \hspace{5mm}\textit{<code block>}                   & kod koji se (u suprotnom) izvršava u svakoj iteraciji                                                 \\                             
        \hline
    \end{tabular}
    \begin{center}
        \footnotesize
        \texttt{continue} služi kako bi se \textbf{preskočilo} izvršavanje \textbf{iteracije}.
    \end{center}

    \section*{\color{NavyBlue} Definiranje funkcije}
    \begin{tabular}{|>{\tt}p{9.00cm}|>{}p{15.50cm}|}
        \hline
        def \textit{<name>}(\textit{<parameters>}):         & definiranje imena i postavljanje parametara (odvojenih zarezom) koji se koriste u bloku koda \\
        \hspace{5mm}\textit{<code block}>                   & blok koda kojeg će funkcija izvršavati svakim pozivanjem                                     \\ 
        \hspace{5mm}return \textit{<value>}                 & povrat (rezultat) funkcije \\ \hline
    \end{tabular}
    \begin{center}
        \footnotesize
        \texttt{return} vraća rezultat funkcije i \textbf{izlazi iz funkcije}, slično kao i \texttt{break} \\
        funkcija ne mora imati \texttt{return} ako ne vraća rezultat, npr. ako radi samo \texttt{print} \\
        funkcija ne mora imati \texttt{parameters} ako nema ulazne podatke. \\
        \texttt{z = f(x,y)} \\
        \texttt{z - \textit{<value>}} \\
        \texttt{f - \textit{<name>}} \\
        \texttt{x,y - \textit{<parameters>}} \\
    \end{center}

    \section*{\color{NavyBlue} Pozivanje funkcije}
    \begin{tabular}{|>{\tt}p{9.00cm}|>{}p{15.50cm}|}
        \hline
        \textit{<name>}(\textit{<arguments>})                           & pozivanje funkcije koja nije imala \texttt{return}                                                        \\ \hline
        \textit{<variable>}=\textit{<name>}(\textit{<arguments>})       & pozivanje funkcije koja je imala \texttt{return} i pohranjivanje njenog rezultata u \texttt{variable}     \\ \hline
    \end{tabular}
    \begin{center}
        \footnotesize
        \texttt{a = f(2,5)} \\
        \texttt{a - \textit{<variable>}} \\
        \texttt{f - \textit{<name>}} \\
        \texttt{2,5 - \textit{<arguments>}} \\
    \end{center}

    \section*{\color{NavyBlue} Parametri funkcije}
    \begin{tabular}{|>{\tt}p{9.00cm}|>{}p{15.50cm}|}
        \hline
        def student(ime, prezime="Horvat", godina=1):                           & definiranje funkcije \texttt{student} s 1 obaveznim i 2 opcionalna parametara \\
        \hspace{5mm}print(ime, prezime, "je", godina, '. godina')               & ispis funkcije \\ \hline
    \end{tabular}
    \begin{center}
        \footnotesize
        Prvo se definiraju svi obavezni parametri, a zatim svi opcionalni parametri.
    \end{center}

    \section*{\color{NavyBlue} Pozicijski argumenti}
    \begin{tabular}{|>{\tt}p{9.00cm}|>{}p{15.50cm}|}
        \hline
        student("Ivan")               & Ivan Horvat je 1. godina \\ \hline
        student("Ivan", "Kovač", 2)   & Ivan Kovač je 2. godina \\ \hline
        student("Ivan", "Kovač")      & Ivan Kovač je 1. godina \\ \hline
        student("Ivan", 2)            & Ivan 2 je 1. godina \\ \hline
    \end{tabular}
    \begin{center}
        \footnotesize
        Pozicijski argumenti zahtjevaju definirani redoslijed i dodjelujuje se s lijevo na desno. \\
        "Nespareni" argumenti dobivaju podrazumjevanu vrijednost.
    \end{center}

    \section*{\color{NavyBlue} Argumenati ključnih riječi}
    \begin{tabular}{|>{\tt}p{9.00cm}|>{}p{15.50cm}|}
        \hline
        student(ime="Ivan")                         & Ivan Horvat je 1. godina \\ \hline
        student(ime="Ivan", godina=2)               & Ivan Horvat je 2. godina \\ \hline
        student(prezime="Kovač", ime="Ivan")        & Ivan Kovač je 1. godina \\ \hline
    \end{tabular}
    \begin{center}
        \footnotesize
        Argumenti ključnih riječi ne zahtjevaju definirani redoslijed. \\
        Nedefinirani argumenti dobivaju podrazumjevanu vrijednost.
    \end{center}

    \section*{\color{NavyBlue} Miješani argumenti}
    \begin{tabular}{|>{\tt}p{9.00cm}|>{}p{15.50cm}|}
        \hline
        student("Ivan", godina=2)                   & Ivan Horvat je 2. godina \\ \hline
        student("Ivan", "Kovač", godina=2)          & Ivan Kovač je 2. godina \\ \hline
    \end{tabular}
    \begin{center}
        \footnotesize
        Pozicijski argumenti se definiraju prije argumenata ključnih riječi. \\
    \end{center}

    \section*{\color{NavyBlue} Primjeri krivog pozivanja funkcija}
    \begin{tabular}{|>{\tt}p{9.00cm}|>{}p{15.50cm}|}
        \hline
        student()                               & pozivanje funkcije bez obaveznih argumenata \\ \hline
        student(ime="Ivan", 2)                  & definiranje argumenta bez ključne riječi nakon onog s ključnom riječi \\ \hline
        student("Ivan", 2, prezime="Kovač")     & dvostruko definiranje argumenta (pozicija 2 i ključna riječ "prezime") \\ \hline
        student(kolegij="Matematika")           & definiranje nepostojećeg parametra \\ \hline
    \end{tabular}

    \section*{\color{NavyBlue} Vidljivost varijabli}
    \begin{itemize}
        \item \textbf{globalne} varijable definirane su u glavnom tijelu programa i vidljive su svim funkcijama
        \begin{itemize}
            \item varijable definirane unutar neke petlje (npr. iteratori) vidljive su i izvan te petlje
            \item varijable definirane u bloku koda unutar grananja vidljive su i izvan grananja
        \end{itemize}
        \item \textbf{lokalne} varijable definirane su unutar neke funkcije i vidljve su toj funkciji i njenim pod-funkcijama
        \begin{itemize}
            \item varijable definirane unutar funkcije mogu se postaviti globalnima korištenjem ključne riječi \texttt{global}
        \end{itemize}
    \end{itemize}
    \begin{tabular}{|>{\tt}p{9.00cm}|>{}p{15.50cm}|}
        \hline
        global \textit{<variable>} & postavljanje \texttt{<variable>} globalnom
        \\
        \textit{<variable>}=\textit{<value>} & definiranje vrijednosti varijable
        \\ \hline
    \end{tabular}
    \begin{center}
        \footnotesize
        Ako je određena varijabla definirana globalno, a zatim i više puta lokalno (rekurzivno) unutar funkcija, njena vrijednost u najunutarnjijoj funkciji imat će "najlokalniju" vidljivu vrijednost. \\
    \end{center}

    \section*{\color{NavyBlue} Liste}
    \begin{tabular}{|>{\tt}p{9.00cm}|>{}p{15.50cm}|}
        \hline
        \textit{list} = [1, 2, 3] & definiranje liste
        \\ \hline
        \textit{list} = [[1, 2, 3], [1, 2, 3], [1, 2, 3]] & definiranje ugniježđene liste
        \\ \hline
    \end{tabular}
    \begin{center}
        \footnotesize
        Liste su \textbf{uređene} strukture podataka što znači da je postoji redoslijed članova. \\
        Članovi liste ne moraju biti isti tipovi podataka. \\
        Liste su identiteno-promijenjive... \\
        \texttt{>> list1 = [12, 9, 3, 7]} \\
        \texttt{>> list2 = list1} \\
        \texttt{>> id(list1) == id(list2)} \\
        \texttt{"True"} \\
        \texttt{>> list1.append(1)} \\
        \texttt{>> list2} \\
        \texttt{[12, 9, 3, 7, 1]} \\
        \texttt{>> id(list1) == id(list2)} \\
        \texttt{"True"} \\
        ... što znači da promjena jedne veže promjenu druge.
    \end{center}

    \section*{\color{NavyBlue} Osnovne operacije s listama}
    \begin{tabular}{|>{\tt}p{9.00cm}|>{}p{15.50cm}|}
        \hline
        \textit{list}[\textit{index}] & indeksiranje lista 
        \\ \hline
        \textit{list}[\textit{[start]}:\textit{[end]}\textit{[:step]}] & komad liste 
        \\ \hline
        \textit{list}[\textit{index}] = \textit{value} & postavljanje nove vrijednosti člana niza
        \\ \hline
        \textit{list}[\textit{[start]}:\textit{[end]}\textit{[:step]}] = \textit{list} & postavljanje nove vrijednosti komada liste s drugom listom (brisanje i umetanje)
        \\ \hline
        [1, 2, 3] + [4, 5, 6] & povezivanje listi (\texttt{[1, 2, 3, 4, 5, 6]})
        \\ \hline
        [1, 2, 3] * 2 & umnožavanje listi (\texttt{[1, 2, 3, 1, 2, 3]})
        \\ \hline
        del \textit{list}[\textit{start}\textit{[:end]}\textit{[:step]}] & briše član ili komad liste
        \\ \hline
        color = [255, 43, 19] & definiranje liste \#...
        \\
        red, green, blue = color & ... i raspakiravanje - pridruživanje po elemantima
        \\ \hline
        item = [4, "Pizza", "Plain", 16.98] & definiranje liste \#\#...
        \\
        quantity, *others, price = item & ... i raspakiravanje - pridruživanje po elemantima
        \\ \hline
    \end{tabular}
    \begin{center}
        \footnotesize
        Posljednji član liste ima indeks -1, predposljednji -2, itd. \\
        \# Lista se može jednostavo rastaviti ako ima jednak broj elemenata. \\
        \#\# Lista se može "složeno" rastaviti tako da jedan element (označen s *) sakuplja sav višak.
    \end{center}

    \section*{\color{NavyBlue} Metode nad listama}
    \begin{tabular}{|>{\tt}p{9.00cm}|>{\tt}p{0.25cm}|>{}p{14.50cm}|}
        \hline
        \textit{list}.append(\textit{object}) & \ding{55} & dodavanje objekta na kraj \textbf{originalne} liste
        \\ \hline
        \textit{list}.extend(\textit{iterable}) & \ding{55} & dodavanje rastavljene iterable na kraj \textbf{originalne} liste
        \\ \hline
        \textit{list}.insert(\textit{index}, \textit{object}) & \ding{55} & dodavanje objekta ispred člana pod indeksom na \textbf{originalnoj} listi 
        \\ \hline
        \textit{list}.clear() & \ding{55} & prazni \textbf{originalnu} listu
        \\ \hline
        \textit{list}.remove(\textit{value}) & \ding{55} & briše prvi član u \textbf{originalnoj} listi koji ima vrijednost \textit{value}
        \\ \hline
        \textit{list}.pop([\textit{index}]) & \ding{51} & uklanja zadnji član u \textbf{originalnoj} listi (član pod indeksom) i vraća uklonjenu vrijednost
        \\ \hline
        \textit{list}.count(\textit{value}) & \ding{51} & ispisuje broj ponavljanja "value"
        \\ \hline
        \textit{list}.reverse() & \ding{55} & invertira \textbf{originalnu} listu
        \\ \hline
        \textit{list}.sort([reverse=True]) & \ding{55} & (naopako) sortira \textbf{originalnu} listu
        \\ \hline
        \textit{list.copy()} & \ding{51} & kopira listu (korisno jer su liste identiteno-promjenljive)
        \\ \hline
    \end{tabular}
    \begin{center}
        \ding{55} - metoda \textbf{nema} povrat tj. \texttt{return} \\
        \ding{51} - metoda \textbf{ima} povrat tj. \texttt{return} \\
    \end{center}

\end{document}