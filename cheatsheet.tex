\documentclass[10pt]{article}

% Paketi koji će se koristiti
\usepackage[a4paper, landscape, margin=2.00cm]{geometry}
\usepackage{amsfonts, amssymb, amsmath}
\usepackage{graphicx}
\usepackage{float}
\usepackage{multirow}
\usepackage{array}
\usepackage[utf8]{inputenc}
\usepackage[T1]{fontenc}
\usepackage{fancyhdr}
\usepackage{lastpage}
\usepackage[dvipsnames]{xcolor}

% Dodaj header i footer
\pagestyle{fancy}
\renewcommand{\headrulewidth}{0pt}
\fancyhf{}
\setlength{\headheight}{32.25pt}
\lhead{}
\rhead{\includegraphics[height=1.00cm]{./vim.png}}
\setlength{\footskip}{12.00pt}
\lfoot{}
\rfoot{str. {\thepage}/{\pageref{LastPage}}}

\begin{document}

\renewcommand{\arraystretch}{1.50}

    % Navigacija
    \section*{\color{ForestGreen} Navigacija}
    \begin{tabular}{|>{\tt}p{9.00cm}|>{}p{15.50cm}|}
        \hline
        h               & lijevo                             \\ \hline
        j               & dolje                              \\ \hline
        k               & gore                               \\ \hline 
        l               & desno                              \\ \hline \hline
        w               & riječ udesno (- interpunkcija)     \\ \hline
        W               & riječ udesno (+ interpunkcija)     \\ \hline
        b               & riječ ulijevo (- interpunkcija)    \\ \hline 
        B               & riječ ulijevo (+ interpunkcija)    \\ \hline 
        e               & kraj riječi (- interpunkcija)      \\ \hline 
        E               & kraj riječi (+ interpunkcija)      \\ \hline \hline
        {\string ^} 0   & početak linije                     \\ \hline
        \$              & kraj linije                        \\ \hline \hline
        (               & početak rečenice                   \\ \hline
        )               & kraj rečenice                      \\ \hline
        \{              & početak odlomka                    \\ \hline
        \}              & kraj odlomka                       \\ \hline \hline
        gg              & prva linija                        \\ \hline
        G               & zadnja linija                      \\ \hline \hline
        ctrl+f          & page up                            \\ \hline
        ctrl+b          & page down                          \\ \hline \hline
        z               & fokus na trenutnu liniju           \\ \hline
    \end{tabular}
    \begin{center}
        \large
        \textcolor{ForestGreen}{\texttt{\textbf{<pomak> = <broj><pokret>}}}
        \\
        ... čime će se \texttt{<pokret>} ponoviti \texttt{<broj>} puta. 
    \end{center}
    \newpage

    % Umetanje
    \section*{\color{ForestGreen} Umetanje}
    \begin{tabular}{|>{\tt}p{9.00cm}|>{}p{15.50cm}|}
        \hline
        i               & umetanje teksta prije kursora                   \\ \hline
        a               & umetanje teksta nakon kursora                   \\ \hline \hline
        I               & umetanje teksta na početku linije               \\ \hline 
        A               & umetanje teksta na kraju linije                 \\ \hline \hline
        o               & umetanje teksta ispod linije                    \\ \hline
        O               & umetanje teksta iznad linije                    \\ \hline \hline
        J               & \textit{append} donje linije na trenutnu liniju \\ \hline
    \end{tabular}
    \begin{center}
        \large
        \textcolor{ForestGreen}{\texttt{\textbf{<broj><naredba><tekst>}}}
        \\
        ... čime će se \texttt{<naredba>} za umetanje (s tekstom) ponoviti \texttt{<broj>} puta. 
    \end{center}

    % Brisanje
    \section*{\color{ForestGreen} Brisanje}
    \begin{tabular}{|>{\tt}p{9.00cm}|>{}p{15.50cm}|}
        \hline
        x       & brisanje znaka na kursoru (\textit{delete})               \\ \hline
        X       & brisanje znaka ispred kursora (\textit{backspace})        \\ \hline \hline
        d       & naredba za brisanje                                       \\ \hline
        dd      & brisanje cijele linije                                    \\ \hline
    \end{tabular}
    \begin{center}
        \large
        \textcolor{ForestGreen}{\texttt{\textbf{<naredba> = <broj><naredba><pokret>}}}
        \\
        ... čime će se \texttt{<naredba>} \textit{delete} ponoviti \texttt{<broj>} puta. 
    \end{center}
    \newpage

    % Kopiranje
    \section*{\color{ForestGreen} Kopiranje}
    \begin{tabular}{|>{\tt}p{9.00cm}|>{}p{15.50cm}|}
        \hline
        y       & naredba za kopiranje                                   \\ \hline
        yy      & kopiranje cijele linije                                \\ \hline \hline
        p       & naredba za kopiranje                                   \\ \hline
        P       & kopiranje cijele linije                                \\ \hline
    \end{tabular}
    \begin{center}
        \large
        \textcolor{ForestGreen}{\texttt{\textbf{<naredba> = <broj><naredba><pokret>}}}
        \\
        ... čime će se \texttt{<naredba>} \textit{yank} ili \textit{put} ponoviti \texttt{<broj>} puta. 
    \end{center}
    \newpage

    % Registri
    \section*{\color{ForestGreen} Registri}
    Sve što je obrisano ili kopirano se sprema u \textit{system-wide} registre.
    \\
    \begin{tabular}{|>{\tt}p{9.00cm}|>{}p{15.50cm}|}
        \hline
        :reg            & pregled registra                                                  \\ \hline \hline
        ""              & neimenovani (default) registar za y, d i x                        \\ \hline
        "0              & registar za y                                                     \\ \hline
        "1 - "9         & registar za d i x                                                 \\ \hline \hline
        "a - "z         & registri za bilošto                                               \\ \hline
        "A - "Z         & registri za \textit{append} u registre \texttt{"a - "z}           \\ \hline
        "\_             & \textit{black hole} registar                                      \\ \hline
        "+ "*           & sistemski \textit{clipboard}                                      \\ \hline
    \end{tabular}
    \begin{center}
        \large
        \textcolor{ForestGreen}{\texttt{\textbf{<registar><naredba>}}}
        \\
        ... čime će se \texttt{<operacija>} (\textit{delete}, \textit{yank}, \textit{put}) spremiti/učitati u/iz registar/a. 
    \end{center}
    \newpage
    
    % Pretraga
    \section*{\color{ForestGreen} Pretraga}
    \subsection*{\color{ForestGreen} Linijske pretrage}
    Kursor se pozicionira na znaku:
    \\
    \begin{tabular}{|>{\tt}p{9.00cm}|>{}p{15.50cm}|}
        \hline
        f<znak>   & pretraga znaka unaprijed            \\ \hline
        F<znak>   & pretraga znaka unazad               \\ \hline
    \end{tabular}
    \\
    Kursor se pozicionira ispred znaka:
    \\
    \begin{tabular}{|>{\tt}p{9.00cm}|>{}p{15.50cm}|}
        \hline
        t<znak>   & pretraga znaka unaprijed            \\ \hline
        T<znak>   & pretraga znaka unazad               \\ \hline
    \end{tabular}
    \\
    Ponavljanje pretrage:
    \\
    \begin{tabular}{|>{\tt}p{9.00cm}|>{}p{15.50cm}|}
        \hline
        ;         & ponavljanje pretrage u istom smjeru         \\ \hline
        ,         & ponavljanje pretrage u suprotnom smjeru     \\ \hline
    \end{tabular}
    \subsection*{\color{ForestGreen} Globalne pretrage}
    Pretraga stringova:
    \\
    \begin{tabular}{|>{\tt}p{9.00cm}|>{}p{15.50cm}|}
        \hline
        /<string>   & pretraga unaprijed            \\ \hline
        ?<string>   & pretraga unazad               \\ \hline
    \end{tabular}
    \\
    Pretraga riječi ispod kursora:
    \\
    \begin{tabular}{|>{\tt}p{9.00cm}|>{}p{15.50cm}|}
        \hline
        *           & pretraga unaprijed        \\ \hline
        \#          & pretraga unazad           \\ \hline
    \end{tabular}
    \\
    Ponavljanje pretrage:
    \\
    \begin{tabular}{|>{\tt}p{9.00cm}|>{}p{15.50cm}|}
        \hline
        n         & ponavljanje pretrage u istom smjeru         \\ \hline
        N         & ponavljanje pretrage u suprotnom smjeru     \\ \hline
    \end{tabular}
    \begin{center}
        \large
        Pretraga funkcionira kao pokret te vrijede sva pravila kombiniranja naredbi!
    \end{center}
    \newpage
    
    % Zamjena
    \section*{\color{ForestGreen} Zamjena}
    \subsection*{\color{ForestGreen} Klasična zamjena}
    \begin{tabular}{|>{\tt}p{9.00cm}|>{}p{15.50cm}|}
        \hline
        r                   & zamjena jednog znaka ispod kursora drugim znakom                                              \\ \hline
        R                   & zamjena znak-po-znak do prekida unosa                                                         \\ \hline                            
        s                   & zamjena jednog znaka ispod kursora s tekstom                                                  \\ \hline \hline
        c                   & naredba za zamjenu                                                                            \\ \hline
        cc                  & zamjena cijele linije                                                                         \\ \hline \hline
        $\sim$              & zamjena znaka \textit{lowercase} $\rightleftarrows$ \textit{uppercase}                        \\ \hline
        g$\sim$             & naredba za zamjenu znaka \textit{lowercase} $\rightleftarrows$ \textit{uppercase}             \\ \hline
        g$\sim$$\sim$       & zamjena reda \textit{lowercase} $\leftrightarrows$ \textit{uppercase}                         \\ \hline \hline
        gu                  & naredba za zamjenu znaka \textit{uppercase} $\rightarrow$ \textit{lowerrcase}                 \\ \hline
        guu                 & zamjena reda \textit{uppercase} $\rightarrow$ \textit{lowercase}                              \\ \hline \hline
        gU                  & naredba za zamjenu znaka \textit{lowercase} $\rightarrow$ \textit{uppercase}                  \\ \hline
        gUU                 & zamjena reda \textit{lowercase} $\rightarrow$ \textit{uppercase}                              \\ \hline
    \end{tabular}
    \begin{center}
        \large
        \textcolor{ForestGreen}{\texttt{\textbf{<naredba> = <broj><naredba><pokret>}}}
        \\
        ... čime će se \texttt{<naredba>} \texttt{c}, \texttt{g$\sim$}, \texttt{gu} ili \texttt{gU} ponoviti \texttt{<broj>} puta. 
    \end{center}
    \subsection*{\color{ForestGreen} \texttt{sed}-like zamjena}
    \begin{center}
        \large
        \textcolor{ForestGreen}{\texttt{\textbf{:<opseg>s/<staro>/<novo>/[g]}}}
        \\
        ... čime se naredba zamjene \texttt{<staro>} za \texttt{<novo>} vrši na \texttt{<opseg>}.
        \\
        \texttt{<staro>} može uključivati i regularne izraze.
        \\
        Ako je uključen \textit{flag} \texttt{g} onda se zamjena radi na svim pojavljivama na liniji, a ne samo prvom.
        \begin{itemize}
            \item \texttt{<opseg>} može biti npr.:
            \begin{itemize}
                \item \texttt{\%} na svim linijima
                \item \texttt{.,\$} od trenutne do zadnje linije
                \item \texttt{1,.} od prve do trenutne linije
                \item \texttt{3} samo na trećoj liniji
                \item \texttt{3,5} od treće do pete linije
            \end{itemize}
        \end{itemize}
    \end{center}
    \newpage

    % Objekti
    \section*{\color{ForestGreen} Objekti}
    Vim prepoznaje pojam "objekata".
    Objekti u Vim-u su:
    \\
    \begin{tabular}{|>{\tt}p{9.00cm}|>{}p{15.50cm}|}
        \hline
        w         & riječ                           \\ \hline
        s         & rečenica                        \\ \hline
        p         & paragraf                        \\ \hline
        ( )       & sadržaj oblih zagrada           \\ \hline
        [ ]       & sadržaj uglatih zagrada         \\ \hline
        \{ \}     & sadržaj vitičastih zagrada      \\ \hline
        < >       & sadržaj izlomljenih zagrada     \\ \hline
        "         & sadržaj navodnika               \\ \hline
        '         & sadržaj navodnika               \\ \hline
        `         & sadržaj navodnika               \\ \hline
        t         & sadržaj \textit{tag}-a          \\ \hline
    \end{tabular}
    \\
    Nad objektima moguće je izvoditi radnje kao što su \textit{delete}, \textit{yank}, \textit{change} i dr. 
    Te radnje se modificiraju modifikatorima:
    \\
    \begin{tabular}{|>{\tt}p{9.00cm}|>{}p{15.50cm}|}
        \hline
        i         & ne uključuje "separator" objekta    \\ \hline
        a         & uključuje "separator" objekata      \\ \hline
    \end{tabular}
    \begin{center}
        \large
        \textcolor{ForestGreen}{\texttt{\textbf{<naredba><modifikator><objekt>}}}
        \\
        ... čime će se izvršiti \texttt{<naredba>} unutar ili nad cijelim \texttt{<objektom>}. 
    \end{center}
    \newpage

    % Macro
    \section*{\color{ForestGreen} Macro}
    Macroi su niz naredbi, a služe za ponavljanje kompleksnih radnji tako što se iste snimaju i pohranjuju u zasebne registre.
    \\
    \begin{tabular}{|>{\tt}p{9.00cm}|>{}p{15.50cm}|}
        \hline
        qa                              & započinje snimanje macroa koji će se pohraniti u registar \texttt{a}                                                      \\ \hline
        q                               & završava snimanje macroa                                                                                                  \\ \hline
        @a                              & izvršava snimljeni macro koji je pohranjen u registar \texttt{a}                                                          \\ \hline
        :<opseg>normal @a               & izvršava snimljeni macro koji je pohranjen u registar \texttt{a} nad opsegom                                              \\ \hline
        "ap                             & \texttt{put} registra u obliku teksta (znakovi za \texttt{esc} i sl. se upisuju u \textit{literal mode}-u nakon Ctrl+v)   \\ \hline
        "add                            & brisanje/pohranjivanje teksta (niza naredbi) u registar                                                                   \\ \hline
        let @a = '<macro>'              & upis \textit{macro}-a u konfiguracijsku datoteku \texttt{.vimrc}                                                          \\ \hline
    \end{tabular}
    \newpage
    
    % Međuspremnici
    \section*{\color{ForestGreen} Međuspremnici}
    Međuspremnici su mjesta u memoriji gdje su učitane Vim datoteke.
    \\
    \begin{tabular}{|>{\tt}p{9.00cm}|>{}p{15.50cm}|}
        \hline
        :buffers :ls                                & otvara listu međuspremnika (otvoreni fileovi i trenutne pozicije kursora)     \\ \hline
        :b 1                                        & otvara međuspremnik 1                                                         \\ \hline
        :b file1.txt                                & otvara file1.txt ako se on nalazi u međuspremniku                             \\ \hline
        :bnext :bn                                  & otvara idući file                                                             \\ \hline
        :bprevious :bp                              & otvara predhodni file                                                         \\ \hline
        :bfirst :bf                                 & otvara prvi file                                                              \\ \hline
        :blast :bl                                  & otvara zadnji file                                                            \\ \hline
        <Ctrl>+{\string ^}                          & otvara ranije otvoreni file                                                   \\ \hline
        :badd file3.txt :ba file3.txt               & dodaje file u međuspremnik                                                    \\ \hline                                                                                                        
        :bdelete [file3.txt] :bd [file3.txt]        & uklanja trenutni file/[file3.txt] iz međuspremnika                            \\ \hline
        :bufdo \%s/<staro>/<novo>/g                 & izvršava naredbu za sve fileove                                               \\ \hline
    \end{tabular}         
    \\
    Međuspremnici imaju posebne "oznake" koji definiraju njihovo stanje učitanosti/prikaza.
    \\
    \begin{tabular}{|>{\tt}p{9.00cm}|>{}p{15.50cm}|}
        \hline
        a                                           & file je učitan i prikazan (active)                                            \\ \hline 
        h                                           & file je učitan i nije prikazan (hidden)                                       \\ \hline                                  
                                                    & file nije učitan i nije prikazan (samo su učitani metapodaci)                 \\ \hline
        \%                                          & file koji se trenutno pregledava                                              \\ \hline
        \#                                          & označava prethodno otvoreni file                                              \\ \hline
        +                                           & označava da je file uređen od zadnjeg spremanja                               \\ \hline
    \end{tabular}         
    \newpage              
    
    % Prozori
    \section*{\color{ForestGreen} Prozori}
    U odvojene prozoru mogu se staviti datoteke koje su učitane u međuspremnike.
    \\
    \begin{tabular}{|>{\tt}p{9.00cm}|>{}p{15.50cm}|}
        \hline
        :split [file] :sp [file] Ctrl+w s           & otvara datoteku u još jednom prozoru (horizontalno)                           \\ \hline
        :vsplit [file] :vs [file] Ctrl+w v          & otvara datoteku u još jednom prozoru (vertikalno)                             \\ \hline
        :ball :ba                                   & otvara sve datoteke iz međuspremnika u svoje prozore                          \\ \hline
        Ctrl+w q                                    & zatvara trenutni prozor                                                       \\ \hline
        :only :on Ctrl+w o                          & zatvara sve prozore osim trenutnog                                            \\ \hline
        :windo \%s/<staro>/<novo>/g                 & izvršava naredbu na svim prozorima                                            \\ \hline
    \end{tabular}         
    \\
    Prozorima se upravlja sljedećim naredbama:
    \\
    \begin{tabular}{|>{\tt}p{9.00cm}|>{}p{15.50cm}|}
        \hline
        Ctrl+w w                                    & prebacuje se u drugi prozor                                                   \\ \hline
        Ctrl+w h                                    & prebacuje se u prozor lijevo                                                  \\ \hline
        Ctrl+w l                                    & prebacuje se u prozor desno                                                   \\ \hline
        Ctrl+w j                                    & prebacuje se u prozor dolje                                                   \\ \hline
        Ctrl+w k                                    & prebacuje se u prozor gore                                                    \\ \hline \hline
        Ctrl+w +                                    & povećava prozor (vertikalno)                                                  \\ \hline
        Ctrl+w -                                    & smanjuje prozor (vertikalno)                                                  \\ \hline
        Ctrl+w >                                    & povećava prozor (horizontalno)                                                \\ \hline
        Ctrl+w <                                    & smanjuje prozor (horizontalno)                                                \\ \hline
        Ctrl+w \_                                   & maksimizira prozor po horizontali                                             \\ \hline
        Ctrl+w |                                    & maksimizira prozor po vertikali                                               \\ \hline
        Ctrl+w =                                    & svi prozoru postaju jednake veličine                                          \\ \hline \hline
        Ctrl+w r                                    & rotira sadržaj prozora (2 $\leftarrow$ 1, 3 $\leftarrow$ 2, 1 $\leftarrow$ 3) \\ \hline
        Ctrl+w R                                    & rotira sadržaj prozora (obrnuti smjer od gornjeg)                             \\ \hline
        Ctrl+w H                                    & pozicionira aktivni prozor lijevo                                             \\ \hline
        Ctrl+w L                                    & pozicionira aktivni prozor desno                                              \\ \hline
        Ctrl+w J                                    & pozicionira aktivni prozor dolje                                              \\ \hline
        Ctrl+w K                                    & pozicionira aktivni prozor gore                                               \\ \hline
    \end{tabular}         
    \newpage              

    % Ostalo
    \section*{\color{ForestGreen} Ostalo}
    \begin{tabular}{|>{\tt}p{9.00cm}|>{}p{15.50cm}|}
        \hline
        :e <datoteka>                   & učitavanja datoteke u međuspremnik                                  \\ \hline
        :r <datoteka>                   & paste sadržaja datoteke na mjestu kursora                           \\ \hline
        :w [<ime>]                      & spremanje promjena [u datoteku pod nazivom ime]                     \\ \hline
        :wq [<ime>]                     & spremanje promjena [u datoteku pod nazivom ime] i izlazak           \\ \hline
        :q!                             & izlazak bez spremanja promjena                                      \\ \hline
        :e!                             & ponovno učitavanje datoteke prije svih promjena                     \\ \hline
        :wall                           & spremanje promjena datoteka iz svih međuspremnika                   \\ \hline
        :wqall                          & spremanje promjena i izlazak iz svih međuspremnika                  \\ \hline
        :qall!                          & izlazak iz svih međuspremnika bez spremanja                         \\ \hline
        :eall!                          & ponovno učitavanje svih međuspremnika prije promjena                \\ \hline
        :h [<naredba>]                  & otvaranje prozora za pomoć [oko naredbe]                            \\ \hline
        :E                              & otvaranje \textit{explorer}-a za datoteke                           \\ \hline
        :f                              & dodatne informacije o trenutnoj datoteci                            \\ \hline
        :! <naredba>                    & izvršavanje naredbe u terminalu pa vraćanje u Vim                   \\ \hline
        :r! <naredba>                   & ispis standardnog izalza naredbe u Vim na mjesto kursora            \\ \hline
        u                               & \textit{undo} svih promjena                                         \\ \hline
        <Ctrl>+r                        & \textit{redo} svih promjena                                         \\ \hline
        .                               & ponavljanje naredbe                                                 \\ \hline
    \end{tabular}         
    \newpage              
\end{document}            